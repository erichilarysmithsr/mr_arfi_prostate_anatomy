\section{Conclusions}
The delineation of prostate zonal anatomy in ARFI images has been compared with
the established methods for identifying zonal anatomy using MR T2W images.
Both imaging modalities showed moderate correlations (0.39 $<$ R$^2 < $ 0.74)
between estimated organ volume and gross pathologic weights ARFI and MR total
prostate gland volumes were well-correlated (R$^2$ = 0.68), but ARFI images
yielded prostate volumes that were, on average, larger (36\% $\pm$ 28\%) than
MR images, primarily due to over-estimation of the anterior-to-posterior
dimension of the prostate total gland (17.0 $\pm$ 12.1\%), while over-estimated
of the other dimensions were less significant contributors (8.1 $\pm$ 18.4\%
and 0.58 $\pm$ 12.9\%).  The central zone volumes of ARFI and MR images were
also moderated correlated (R$^2$ = 0.41), with minimal volume bias between the
imaging modalities, but significant variability case-to-case (2.1 $\pm$
39.1\%).  Central zone volume differences were, again, strongly attributed to
over-estimation of the anterior-to-posterior axis (14.8 $\pm$ 23.1\%), with a
significant underestimation of the apex-to-base dimension (-10.8 $\pm$ 22.3\%)
and no mean bias in the lateral-to-lateral measurements (0.006 $\pm$ 17.2\%).
Strong variability in central gland volumes is believed to be related to the
extent of benign prostatic hyperplasia (BPH) for select cases.  Overall, ARFI
imaging of the prostate yielded prostate volumes and dimensions that were
correlated with MR T2WI estimates, with biases in the anterior-to-posterior
dimension, most likely related to poor displacement SNR in the anterior region
of the prostate from greater distance from the rectal wall imaging surface.
ARFI imaging is a promising low-cost, real-time imaging modality that can
compliment MR imaging for diagnosis, treatment planning and management of PCa.
