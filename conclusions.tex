\section{Conclusions}
The delineation of prostate zonal anatomy in ARFI images has been compared with
the established methods for identifying zonal anatomy using MR T2W images.
Both imaging modalities showed moderate correlations between estimated organ
volume and gross pathologic weights, and ARFI:MR total prostate gland volumes
were well-correlated (R$^2$ = 0.63), but ARFI images yielded prostate volumes
that were, on average, larger (6.1\% $\pm$ 25\%) than MR images, primarily due
to over-estimation of the lateral dimension of the prostate total gland
(\ARFImrTotalLatLatMeanPct~ $\pm$ \ARFImrTotalLatLatStdPct\%), while
over-estimates of the other dimensions were less significant contributors
(\ARFImrTotalAntPostMeanPct~$\pm$ \ARFImrTotalAntPostStdPct\%~and
\ARFImrTotalApexBaseMeanPct~$\pm$ \ARFImrTotalApexBaseStdPct\%).  The central
zone volumes of ARFI and MR images were also moderately correlated (R$^2$ =
0.38), with minimal volume bias between the imaging modalities, but significant
variability case-to-case (-5.0 $\pm$ 39.5\%).  Central zone volume differences
were, again, strongly attributed to over-estimation of the lateral dimension
(\ARFImrCentralLatLatMeanPct~$\pm$ \ARFImrCentralLatLatStdPct\%), with a
significant underestimation of the anterior-to-posterior dimension
(\ARFImrCentralAntPostMeanPct~$\pm$ \ARFImrCentralAntPostStdPct\%).  Strong
variability in central gland volumes is believed to be related to the extent of
benign prostatic hyperplasia (BPH) for select cases.  Overall, ARFI imaging of
the prostate yielded prostate volumes and dimensions that were correlated with
MR T2WI estimates, with biases in the anterior-to-posterior dimension, most
likely related to poor displacement SNR in the anterior region of the prostate
from greater distance from the rectal wall imaging surface, and the lateral
dimension, where contrast between the PZ and peri-prostatic fat could be
limited.  ARFI imaging is a promising low-cost, real-time imaging modality that
can compliment MR imaging for diagnosis, treatment planning and management of
PCa.
