\section{Conclusions}
The delineation of prostate zonal anatomy in \textbf{B-mode/}ARFI images has
been compared with the established methods for identifying zonal anatomy using
MR T2W images.  Both imaging modalities showed moderate correlations between
estimated organ volume and gross pathologic weights, and \textbf{B-mode}:MR
total prostate gland volumes were well-correlated (R$^2$ = \MRarfiVolTotalRsq),
but ARFI images yielded prostate volumes that were, on average, larger
(\MRarfiVolTotalMeanDiff\%~$\pm$ \MRarfiVolTotalStdDiff\%) than MR images,
primarily due to over-estimation of the lateral dimension of the prostate total
gland (\ARFImrTotalLatLatMeanPct~ $\pm$ \ARFImrTotalLatLatStdPct\%), while
differences in the other dimensions were less significant contributors
(\ARFImrTotalAntPostMeanPct~$\pm$ \ARFImrTotalAntPostStdPct\%~and
\ARFImrTotalApexBaseMeanPct~$\pm$ \ARFImrTotalApexBaseStdPct\%).  The central
gland volumes of ARFI and MR images were also well-correlated (R$^2$ =
\MRarfiVolCentralRsq), with minimal volume bias between the imaging modalities,
but significant variability case-to-case (\MRarfiVolCentralMeanDiff~$\pm$
\MRarfiVolCentralStdDiff\%).  Central gland volume differences were, again,
strongly attributed to over-estimation of the lateral dimension
(\ARFImrCentralLatLatMeanPct~$\pm$ \ARFImrCentralLatLatStdPct\%), with a
significant underestimation of the apex-to-base dimension
(\ARFImrCentralAntPostMeanPct~$\pm$ \ARFImrCentralAntPostStdPct\%).  Strong
variability in CG volumes is believed to be related to the extent of BPH for
select cases.  Overall, B-mode/ARFI imaging of the prostate yielded prostate
volumes and dimensions that were correlated with MR T2WI estimates, with
differences in the lateral and apex-to-base dimensions.  Poor image contrast
between the prostate and extraprostatic fat may be one cause of these
differences.  \textbf{B-mode imaging combined with} ARFI imaging is a promising
low-cost, real-time imaging modality that can complement MR imaging for
diagnosis, treatment planning and management of PCa.
