\section{Conclusions}
The delineation of prostate zonal anatomy in ARFI images has been compared with
the established methods for identifying zonal anatomy using T2 MR images.  In
XX cases of prostates containing varying degrees of PCa and BPH, perpipheral
zone volumes\ldots XXXXX and central gland volumes\ldots XXXXX.  Aspect ratios
of the central glad agreed to within XX\% between MR and ARFI imaging datasets.
Appreciable amounts of BPH made determining the transition between the central
gland and peripheral zone difficult to discern and, these cases were not
included in this analysis.  Additionally, large PCa lesions can also distort
prostate zonal anatomy appreciably, making the distinction between peripheral
zone and central gland challenging.  Overall, ARFI imaging is able to delineate
central gland from peripheral zone in the prostate in the absence of excessive
BPH or PCa tumor burden.
