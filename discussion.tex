\section{Discussion}\label{sect:discussion}
This study demonstrated that ARFI imaging delineates zonal anatomy of the
prostate well when compared to MR T2WI, with comparable estimates of total and
central gland volume (Figure~\ref{fig:mr_arfi_volumes}).  Establishing accurate
delineation of the prostate's zonal anatomy is important since PCa has a strong
correlations with location in the prostate, specifically in the peripheral
zone, and accurate capsule delineation will facilitate future efforts to
register 3D imaging datasets between ARFI and MR imaging.  

This was the first quantitative comparison between ARFI imaging and a newly
established clinical imaging modality to evaluate the prostate for PCa
detection and characterization, and we have drawn some useful conclusions in
how our zonal anatomy boundaries are determined during image segmentation.
Poor contrast between the PZ and peri-prostatic fat in B-mode images can lead
to an overestimation of lateral boundaries of the prostate in our ultrasound
datasets compared to MR T2WI, leading to an overall overestimation of prostate
total gland volume.  The anterior aspect of the prostate can also be
challenging to delineate, especially in large prostates, where SNR decreases
with increasing depth away from our rectal wall imaging surface and MR images
suffer from gland heterogeneity in this region~\cite{Gupta2013}, though we had
good agreement between ARFI and MR T2W images in the anterior-to-posterior
dimension (Figure~\ref{fig:mr_arfi_path_axes}(b),
Table~\ref{tab:mr_arfi_axes_error}).  Surprisingly, ARFI imaging consistently
underestimated the extent of the prostate capsule and CG along the apex-to-base
axis relative to MR T2WI (Figure~\ref{fig:mr_arfi_path_axes}(c),
Table~\ref{tab:mr_arfi_axes_error}), though the reasons for this
underestimation are not clear.

This study has several limitations that should be considered when interpreting
these results.  Gross pathology weight and axis measurements could both be
affected by the presence of peri-prostatic tissue that was excised during
radical prostatectomy, especially in cases where more aggressive margins may
have been necessary.  For this reason, unlike the image-to-image measurement
comparisons, all of the image metrics presented relative to pathology metrics
(Figure~\ref{fig:mr_arfi_weight}) were not characterized for absolute accuracy,
but instead, relative correlations were evaluated.  Additionally, the volumes
of the prostate from gross pathologic measurements were approximated as
ellipsoids, which also introduced error, most likely an over-estimation of
volume.  Interestingly, all pathology estimates were thought to have positive
biases, but both imaging modalities tended to overestimate volume relative to the
pathology measurements.

It should also be noted that all of the prostates in this study contained
varying amounts of PCa, BPH and atrophy, all of which can distort the zonal
anatomy, especially in the case of BPH and central gland morphology.  While
younger, healthier prostates could have been targeted, these healthy organs
would not have been excised for pathology characterization, and the zonal
anatomy of a healthy (young) prostate is expected to be different from the
prostate of a middle-age man, who is the target demographic for PCa screening
imaging and PCa characterization. 

This study did not evaluate user biases in image segmentation.  While MR zonal
anatomy delineation has some establishment in the clinical literature, this
study is the first attempt to define the criteria for ARFI imaging zonal
anatomy characteristics, and it is expected that such delineations will
continue to be refined as we acquire more cases and continue to compare with
MRI and pathology data.  Given this limitation, no attempts were made to
further quantify reader-to-reader variability in this work.

Future directions for our research include deeper analysis of our findings,
especially the lateral dimension over-estimation, to improve our anatomic
delineation and assess further its clinical impact.  Technological improvements
to improve anterior prostate boundary delineation in both B-mode and ARFI
imaging are also being pursued and will hopefully allow for the currently
elusive anterior PCa lesions in MR to be seen in ARFI imaging.  3D models of
the prostate and central glands will be used across ARFI and MR imaging
datasets to spatially register them to facilitate correlation of regions of PCa
suspicion between the two modalities, and to correlate with whole-mount
histology of the excised prostate specimens.  This effort will allow the
sensitivity and specificity of each imaging modality for PCa detection and
characterization to be quantified.  While ARFI imaging of the prostate in a
constantly evolving technology, the fact that one can view the different
prostatic zones on ARFI images in addition to MRI is encouraging and could lead
to targeted diagnostic biopsies and therapies with real-time imaging for men
with PCa. 
