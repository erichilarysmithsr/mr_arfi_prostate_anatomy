\section{Discussion}\label{sect:discussion}

This study was designed to compared the zonal anatomy of the prostate in
patients with biopsy-proven PCa as seen in ARFI imaging and MR T2WI.  Overall,
ARFI imaging performed well when compared with MR estimates of total and
central gland volume (Figure~\ref{fig:mr_arfi_volumes}), though ARFI tended to
overestimate prostate total gland volume, most commonly due to over-estimation
of the anterior-to-posterior dimension of the prostate.  As seen in Figure XX,
clear delineation of the anterior aspect of the prostate can be challenging in
ARFI and B-mode images due to poor SNR, since this is the deepest aspect of the
prostate relative to the rectal wall imaging surface.  MR T2WI can also have
difficulties visualizing and diagnosing lesions in the anterior region of the
prostate,~\cite{Gupta2013} and was evident in imaging study subject 4, with a
$\sim$143 g prostate (Figure~\ref{fig:mr_arfi_weight}).  With study subject 4
excluded, MR and ARFI imaging volumes and the ellipsoidal prostate volume
estimates had similar correlation with prostate weights
(Figure~\ref{fig:mr_arfi_weight}(d)).

The most accurate measurements for both the total prostate gland and central
gland were in the lateral-to-lateral dimension
(Figure~\ref{fig:mr_arfi_path_axes}(a), Table~\ref{tab:mr_arfi_axes_error}),
with ARFI imaging exhibiting a slight overestimate in both glands.  ARFI
imaging had weaker correlations with MR imaging in the apex-to-base dimension
(Figure~\ref{fig:mr_arfi_path_axes}(c)), though it should be noted that the MR
images are limited in spatial resolution by slice thickness in this dimension
($\sim$ 3 mm).  Finally, as expected from qualitative image evaluation, the
anterior-to-posterior dimension had the weakest correlations, especially in
the central gland, most likely related to the inability to clearly resolve an
anterior border for both the prostate total and central glands.

This study has several limitations that should be considered when interpreting
these results.  Gross pathology weight and axis measurements both could be
affected by the presence of periprostatic tissue that was excised during
radical prostatectomy, especially in cases where more aggressive margins may
have been necessary.  For this reason, unlike the image-to-image measurement
comparisons, all of the image metrics presented relative to pathology metrics
(Figure~\ref{fig:mr_arfi_weight}) were not characterized for absolute accuracy,
but instead, relative correlations were evaluated.  Additionally, the volumes
of the prostate from gross pathologic measurements were approximated as
ellipsoids, which also introduced error, most likely an over-estimation of
volume.  Interestingly, all pathology estimates were thought to have positive
biases, but both imaging modalities tended to overestimate relative to the
pathology measurements.

It should also be noted that all of the prostates in this study contained
varying amounts of PCa, BPH and atrophy, all of which can distort the zonal
anatomy, especially in the case of BPH and central gland morphology.  While
younger, healthier prostates could have been targeted, these healthy organs
would not have been excised for pathology characterization, and the zonal
anatomy of a health (young) prostate is expected to be different from the
prostate of a middle-age man, who is the target demographic for PCa screening
imaging and PCa characterization; therefore, it was felt that performing this
analysis in the presence of these confounding factors was appropriate.
