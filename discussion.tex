\section{Discussion}\label{sect:discussion}
This study was designed to compared the zonal anatomy of the prostate in
patients with biopsy-proven PCa as seen in ARFI imaging and MR T2WI.  Overall,
ARFI imaging performed well when compared with MR estimates of total and
central gland volume (Figure~\ref{fig:mr_arfi_volumes}), though ARFI tended to
overestimate prostate total gland volume, most commonly due to over-estimation
of the lateral dimension of the prostate.  This can be due to poor contrast
between the PZ and peri-prostatic fat (Figure~\ref{fig:arfi_segs}).
Figure~\ref{fig:arfi_segs} also demonstrates challenges in delineating the
anterior aspect of the prostate in both ARFI and B-mode images due to poor SNR
since this is the deepest aspect of the prostate relative to the rectal wall
imaging surface. MR T2WI can also have difficulties visualizing and diagnosing
lesions in the anterior region of the prostate,~\cite{Gupta2013}, which is
thought to be secondary to the heterogeneity of the gland in that area,
especially in the presence of BPH.  This was evident in imaging study subject
4, with a $\sim$143 g prostate (Figure~\ref{fig:mr_arfi_weight}), which was an
outlier in terms of weight and volume.  While the anterior aspect of the
prostate can be challenging to visualize, it was the axis with the best
agreement between ARFI and MR imaging (Figure~\ref{fig:mr_arfi_path_axes}(b),
Table~\ref{tab:mr_arfi_axes_error}).  ARFI images consistently underestimated
the extent of the prostate capsule and CG along the apex-to-base axis relative
to MR T2WI (Figure~\ref{fig:mr_arfi_path_axes}(c),
Table~\ref{tab:mr_arfi_axes_error}), though the reasons for this
underestimation were not clear.

This study has several limitations that should be considered when interpreting
these results.  Gross pathology weight and axis measurements could both be
affected by the presence of periprostatic tissue that was excised during
radical prostatectomy, especially in cases where more aggressive margins may
have been necessary.  For this reason, unlike the image-to-image measurement
comparisons, all of the image metrics presented relative to pathology metrics
(Figure~\ref{fig:mr_arfi_weight}) were not characterized for absolute accuracy,
but instead, relative correlations were evaluated.  Additionally, the volumes
of the prostate from gross pathologic measurements were approximated as
ellipsoids, which also introduced error, most likely an over-estimation of
volume.  Interestingly, all pathology estimates were thought to have positive
biases, but both imaging modalities tended to overestimate volume relative to the
pathology measurements.

It should also be noted that all of the prostates in this study contained
varying amounts of PCa, BPH and atrophy, all of which can distort the zonal
anatomy, especially in the case of BPH and central gland morphology.  While
younger, healthier prostates could have been targeted, these healthy organs
would not have been excised for pathology characterization, and the zonal
anatomy of a healthy (young) prostate is expected to be different from the
prostate of a middle-age man, who is the target demographic for PCa screening
imaging and PCa characterization. 

It should also be noted that this study did not evaluate user biases in image
segmentation.  While MR zonal anatomy delineation has some establishment in the
clinical literature, this study is the first attempt to define the criteria for
ARFI imaging zonal anatomy characteristics, and it is expected that such
delineations will continue to be refined as we acquire more cases and continue
to compare with MRI and pathology data.  Given this limitation, no attempts
were made to further quantify reader-to-reader variability in this work.

Future directions for our research include deeper analysis of our findings,
especially the lateral dimension over-estimation, to improve our anatomic
delineation and assess further its clinical impact.  Technological improvements
to improve anterior prostate boundary delineation in both B-mode and ARFI
imaging are also being pursued and will hopefully allow for the currently
elusive anterior PCa lesions in MR to be seen in ARFI imaging.  Although work
remains to be done, the fact that one can view the different prostatic zones on
ARFI images in addition to MRI is encouraging and could lead to targeted
diagnostic biopsies and therapies with real-time imaging for men with PCa. 
