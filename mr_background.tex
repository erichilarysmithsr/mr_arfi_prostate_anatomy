\subsection{MR Imaging in Prostate Diagnostics}
MR imaging has been available for use in the workup of patients with PCa since
the early 1980s, but the early work on its diagnostic accuracy is heterogeneous
due to heavy reliance on morphology via T1 and T2-weighted imaging (T2WI). More
recent abilities to include not only anatomic, but also biologic and functional
dynamic parameters, into MR analysis via diffusion-weighted imaging (DWI),
dynamic contrast-enhanced (DCE) imaging or MR spectroscopic imaging (MRSI) is
promising in the future diagnosis and management of PCa.

Currently prostate MR focuses on a multiparametric approach, where two-or-more
imaging sequences, including anatomic and functional data, are used together to
make a diagnosis.~\cite{Barentsz2012} As MR technology continues to evolve and
improve, its role in PCa diagnosis, staging, treatment planning and follow-up
has gained much attention.

\subsubsection{T2-Weighted Imaging and Prostate Anatomy}
T2WI sequences are crucial components of prostate MR imaging.  T2WI is
particularly useful in prostate analysis due to its excellent soft tissue
contrast resolution, which can be maximized by using thin sections of 3--4 mm
and a small field of view of approximately 14 cm.~\cite{Gupta2013,Bonekamp2011}
T2 sequences are the most helpful for tumor localization, as they can clearly
show overall prostate morphology, internal structures and prostatic
margins.~\cite{Gupta2013}

The prostate can be divided into glandular and non-glandular components. The
glandular components include the peripheral zone (PZ) and the central gland,
which are typically easily distinguishable on T2WI. The central gland includes
the central zone, transition zone and the periurethral glandular
tissue.~\cite{Jung2012} Other anatomical markers such as the urethra,
verumontanum and ejaculatory are also often seen on T2WI. 

Approximately 70\% of the prostatic tissue is found in the PZ, which is high in
water content, and thus has higher signal intensity in T2WI.~\cite{Jung2012}
Seventy-five percent of prostatic tumors are found in the PZ and normally show
hypointense T2 signal when compared to the higher intensity
PZ.~\cite{Hricak2007,Hegde2013}; however, tumors can sometimes have similar
intensity as the surrounding tissue and false positives can occur secondary to
post-biopsy changes/hemorrhage, hyperplasia or prostatitis, making diagnosis
more challenging.~\cite{Hegde2013}

\subsubsection{Functional MR Sequences}
Even though T2WI is the mainstay of prostate MR, its overall performance in
prostate cancer diagnosis is not optimal. The incorporation of two or more
functional sequences in multiparametric MR imaging (mpMRI) has been shown to
significantly improve the performance of MRI in cancer
diagnosis.~\cite{Turkbey2012}  The European Society of Urogenital Radiology
(ESUR) prostate MR guidelines recommend at least 2 functional imaging
techniques, in addition to T2WI, to better characterize prostate
tumors.~\cite{Barentsz2012} In a study by Turkbey \etal, researchers found that
mpMRI had a PCa detection positive predictive value of 98\%.~\cite{Turkbey2012}
Functional sequences include diffusion-weighted imaging (DWI), dynamic contrast
enhanced imaging (DCE-MRI) and MR spectroscopic imaging (MRSI).  While mpMRI is
critical in improving PCa detection, T2WI is mainstay of delineating prostate
anatomy, and for the purposes of comparing prostate anatomy between ARFI and MR
images, T2WI were used exclusively in this study.
