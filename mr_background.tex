Clinical Background and Significance 
Prostate cancer (PCa) is the most common noncutaneous malignancy among men in the United States. Approximately 1 in every 6 men will develop PCa during their lifetime, with the median age of diagnosis at 67. 1It is also the second leading cause of cancer related death, with 1 in 36 men dying from the disease. The National Cancer Institute estimates that 238,590 men will be diagnosed with PCa in 2013 and 29,720 will die from the disease.1

PCa diagnosis usually begins by screening with prostate specific antigen (PSA) and digital rectal examination (DRE).  Definitive diagnosis is made by random transrectal ultrasonography-guided (TRUS) biopsies, which are then used to provide the clinician with the proper Gleason score. The combination of these factors—as well as staging—determines the appropriate therapy and prognosis. 

PCa screening has led to earlier diagnosis of smaller tumors and more localized disease.  However, it is well known that the sensitivity and specificity of PSA and DRE are not optimal. In addition, DRE has a low predictive value at lower PSA ranges, and PSA yields many false positives.2-4 As such, a theoretical risk of over-diagnosis and treatment of low-grade—and possibly clinically insignificant—disease exists. Moreover, due to the random nature of biopsies, cancer located outside the routine sampling site can be missed and the extent of the cancer might be underestimated.3 5 For example, in a study by Mufarrij et al 45.9%-47.2% of patients who were candidates for active surveillance but underwent radical prostatectomy had a higher Gleason score on final histopathology than after TRUS biopsy.6 These inaccuracies may lead to inappropriate diagnosis, imprecise risk assessment and potentially avoidable morbidity.

The Use of Magnetic Resonance Imaging in Prostate Diagnostics
Magnetic resonance (MR) imaging has been available for use in the workup of patients with PCa since the early 1980s but the early work on its diagnostic accuracy is heterogeneous.7 Earlier MR techniques relied mostly on morphology via T1 and T2- weighted imaging (T2WI). The more recent ability to include not only anatomic but also biologic and functional dynamic parameters into MR analysis—via diffusion-weighted imaging (DWI), dynamic contrast-enhanced (DCE) imaging or MR spectroscopic imaging (MRSI)—is promising in the future diagnosis and management of PCA.

 Currently prostate MR focuses on a multiparametric approach, where 2 or more imaging sequences—including anatomic and functional data—are used together to try to arrive to a diagnosis.8 As MR technology continues to evolve and improve, its role in PCA diagnosis, staging, treatment planning and follow-up has gained much attention.

T2-Weighted Imaging and Prostate Anatomy
T2WI sequences are crucial components of prostate MR imaging.  T2WI is particularly useful in prostate analysis due to its excellent soft tissue contrast resolution, which can be maximized by using thin sections of 3-4mm and a small field of view of approximately 14cm.3,9 T2 sequences are the most helpful for tumor localization as they can clearly show overall prostate morphology, internal structures and prostatic margins.3

The prostate can be divided into glandular and nonglandular components. The glandular components include the peripheral zone (PZ) and the central gland, which are typically easily distinguishable on T2WI. The central gland includes the central zone, transition zone and the periurethral glandular tissue.10 Other anatomical markers such as the urethra, verumontanum and ejaculatory are also often seen on T2WI. 

Approximately 70% of the prostatic tissue is found in the peripheral zone, which is high in water content and thus of higher signal intensity in T2WI.10 Seventy five percent of prostatic tumors are found in the PZ and normally show hypointense T2 signal when compared to the higher intensity PZ.4 11 However tumors can sometimes seem of similar intensity as the surrounding tissue and false positives can occur secondary to post biopsy changes/hemorrhage, hyperplasia or prostatitis, making diagnosis more challenging.11 

Functional MR Sequences
Even though T2WI is the mainstay of prostate MR, its overall performance in prostate cancer diagnosis is not optimal. The incorporation of two or more functional sequences in multiparametric MR imaging (mpMRI) has been shown to significantly improve the performance of MRI in cancer diagnosis.12 In fact, the European Society of Urogenital Radiology’s (ESUR) prostate MR guidelines recommend at least 2 functional imaging techniques in addition to T2WI in order to better characterize prostate tumors.8 In a study by Turkbey et al, researchers found that mpMRI had a positive predictive value of 98% in prostate cancer detection.12 Functional sequences include diffusion- weighted imaging (DWI), dynamic contrast enhanced imaging (DCE-MRI) and MR spectroscopic imaging (MRSI). 

Diffusion-weighted imaging
DWI is based on the free movement of water particles in tissue and measures the degree of motion restriction.13 Normal prostatic tissue is very glandular with plenty of water molecule movement. On the other hand, tumors have high cellular density and restricted water movement, which leads to decreased diffusion. Due to this restricted diffusion, tumors seem to be of higher intensity on DWI.13

DWI sequences can be processed to obtain apparent diffusion coefficient (ADC) maps. These serve as a more objective measure of diffusivity since they are independent of magnetic field and thus overcome the effects of T2 shine-through.13,14  ADC maps are also used to visually assess the tumor, which appears as a focus of decreased signal intensity when compared to normal prostate tissue.14 

In addition to knowing the location of the cancer within the prostate, being able to identify which cancers will exhibit more aggressive behavior is important when making clinical management decisions. A number of studies have found that lower ADC values correlate with tumors that have higher Gleason scores. 15-17 This could be explained by the fact that higher-grade lesions are usually of higher cellular density and thus have even more marked restricted diffusion. This inverse correlation suggests that ADC maps might be helpful when characterizing lesions and assessing tumor aggressiveness.18 

Dynamic contrast enhanced imaging
DCE-MRI has been shown to have a high sensitivity in cancer detection, especially when combined with other MR imaging modalities.19-22 In a study by Hara et al, this functional modality identified 93% of clinically significant cancers.23 DCE-MRI provides a measure of tumor vascularity and takes advantage of tumor-induced angiogenesis in PCa. 24Increased vascularity in PCa results in earlier and higher peak contrast enhancement and a faster washout when compared to normal prostatic tissue. These known characteristics of PCa are helpful when trying to localize lesions. 

A variety of different approaches can be used to analyze DCE-MRI and no report has, up to date, described the superiority of one method over the others.3 The approaches include qualitative analysis by visual assessment of the images, semi-quantitative methods that look at kinetic parameters on a voxel-by-voxel basis, or quantitative methods that that convert signal intensity into kinetic parameters to determine how much contrast is being exchanged between the vascular, extravascular and extracellular spaces.3
	
MR spectroscopy
Like DWI, MRSI can be helpful in lesion characterization as it provides information on the presence of certain metabolites in tissue. 8Choline and citrate are especially useful in the setting of PCa. Choline is critical in cell membrane synthesis and is thus usually elevated in cells with cancerous behavior.3 Healthy prostate epithelium synthesizes and secretes large quantities of citrate but levels are decreased in PCa.25,26 Therefore, an increase in the choline-to-citrate ratio on MRSI can be used as an indicator of malignancy.9 

Although some data suggests that MRSI increases the accuracy of tumor volume detection and staging when used in addition to anatomic imaging, it is unclear whether it is superior to other modalities.9 ESUR lists MSRI as an optional modality for the diagnosis of PCa since it significantly lengthens exam time and requires specific technical expertise.8 Thus, it is not surprising that many academic centers in the United States do not include MSRI in their mpMRI prostate cancer protocols.3



1.	Howlader N, Noone AM, Krapcho M, Neyman N, Aminou R, Waldron W. SEER Cancer Statistics Review, 1975–2010, National Cancer Institute. Bethesda, MD, based on November 2012 SEER data submission, posted to the SEER web site, 2013. http://seer.cancer.gov/csr/1975_2010 (Accessed on June 08, 2013). 2011.
2.	Gosselaar C, Roobol MJ, Roemeling S, van der Kwast TH, Schröder FH. Screening for prostate cancer at low PSA range: the impact of digital rectal examination on tumor incidence and tumor characteristics. The Prostate. 2007;67(2):154-161 %@ 1097-0045.
3.	Gupta RT, Kauffman CR, Polascik TJ, Taneja SS, Rosenkrantz AB. The state of prostate MRI in 2013. Oncology. Apr 2013;27(4):262-270.
4.	Hricak H, Choyke PL, Eberhardt SC, Leibel SA, Scardino PT. Imaging prostate cancer: A multidisciplinary perspective1. Radiology. 2007;243(1):28-53 %@ 0033-8419.
5.	Cornud F, Delongchamps NB, Mozer P, et al. Value of Multiparametric MRI in the Work-up of Prostate Cancer. Current urology reports. 2012;13(1):82-92 %@ 1527-2737.
6.	Mufarrij P, Sankin A, Godoy G, Lepor H. Pathologic outcomes of candidates for active surveillance undergoing radical prostatectomy. Urology. 2010;76(3):689-692 %@ 0090-4295.
7.	Poon PY, McCallum RW, Henkelman MM, et al. Magnetic resonance imaging of the prostate. Radiology. 1985;154(1):143-149 %@ 0033-8419.
8.	Barentsz JO, Richenberg J, Clements R, et al. ESUR prostate MR guidelines 2012. European radiology. 2012;22(4):746-757 %@ 0938-7994.
9.	Bonekamp D, Jacobs MA, El-Khouli R, Stoianovici D, Macura KJ. Advancements in MR imaging of the prostate: from diagnosis to interventions. Radiographics. 2011;31(3):677-703 %@ 0271-5333.
10.	Jung AJ, Westphalen AC. Imaging Prostate Cancer. Radiologic Clinics of North America. 2012;50(6):1043-1059 %@ 0033-8389.
11.	Hegde JV, Mulkern RV, Panych LP, et al. Multiparametric MRI of prostate cancer: An update on state‐of‐the‐art techniques and their performance in detecting and localizing prostate cancer. Journal of Magnetic Resonance Imaging. 2013;37(5):1035-1054 %@ 1522-2586.
12.	Turkbey B, Choyke PL. Multiparametric MRI and prostate cancer diagnosis and risk stratification. Current opinion in urology. 2012;22(4):310-315 %@ 0963-0643.
13.	Koh D-M, Collins DJ. Diffusion-weighted MRI in the body: applications and challenges in oncology. American Journal of Roentgenology. 2007;188(6):1622-1635 %@ 0361-1803X.
14.	Tan CH, Wang J, Kundra V. Diffusion weighted imaging in prostate cancer. European radiology. 2011;21(3):593-603 %@ 0938-7994.
15.	Kobus T, Vos PC, Hambrock T, et al. Prostate cancer aggressiveness: in vivo assessment of MR spectroscopy and diffusion-weighted imaging at 3 T. Radiology. 2012;265(2):457-467 %@ 0033-8419.
16.	Woodfield CA, Tung GA, Grand DJ, Pezzullo JA, Machan JT, Renzulli JF. Diffusion-weighted MRI of peripheral zone prostate cancer: comparison of tumor apparent diffusion coefficient with Gleason score and percentage of tumor on core biopsy. American Journal of Roentgenology. 2010;194(4):W316-W322 %@ 0361-0803X.
17.	Hambrock T, Somford DM, Huisman HJ, et al. Relationship between apparent diffusion coefficients at 3.0-T MR imaging and Gleason grade in peripheral zone prostate cancer. Radiology. 2011;259(2):453-461 %@ 0033-8419.
18.	Vargas HA, Akin O, Franiel T, et al. Diffusion-weighted endorectal MR imaging at 3 T for prostate cancer: tumor detection and assessment of aggressiveness. Radiology. 2011;259(3):775-784 %@ 0033-8419.
19.	Tanimoto A, Nakashima J, Kohno H, Shinmoto H, Kuribayashi S. Prostate cancer screening: The clinical value of diffusion‐weighted imaging and dynamic MR imaging in combination with T2‐weighted imaging. Journal of Magnetic Resonance Imaging. 2007;25(1):146-152 %@ 1522-2586.
20.	Girouin N, Mège-Lechevallier F, Senes AT, et al. Prostate dynamic contrast-enhanced MRI with simple visual diagnostic criteria: is it reasonable? European radiology. 2007;17(6):1498-1509 %@ 0938-7994.
21.	Ocak I, Bernardo M, Metzger G, et al. Dynamic contrast-enhanced MRI of prostate cancer at 3 T: a study of pharmacokinetic parameters. American Journal of Roentgenology. 2007;189(4):W192-W201 %@ 0361-0803X.
22.	Kim JK, Hong SS, Choi YJ, et al. Wash‐in rate on the basis of dynamic contrast‐enhanced MRI: Usefulness for prostate cancer detection and localization. Journal of Magnetic Resonance Imaging. 2005;22(5):639-646 %@ 1522-2586.
23.	Hara N, Okuizumi M, Koike H, Kawaguchi M, Bilim V. Dynamic contrast‐enhanced magnetic resonance imaging (DCE‐MRI) is a useful modality for the precise detection and staging of early prostate cancer. The Prostate. 2005;62(2):140-147 %@ 1097-0045.
24.	Noworolski SM, Henry RG, Vigneron DB, Kurhanewicz J. Dynamic contrast‐enhanced MRI in normal and abnormal prostate tissues as defined by biopsy, MRI, and 3D MRSI. Magnetic resonance in medicine. 2005;53(2):249-255 %@ 1522-2594.
25.	Costello LC, Franklin RBa, Narayan P. Citrate in the diagnosis of prostate cancer. The prostate. 1999;38(3):237-245 %@ 1097-0045.
26.	Kurhanewicz J, Swanson MG, Nelson SJ, Vigneron DB. Combined magnetic resonance imaging and spectroscopic imaging approach to molecular imaging of prostate cancer. Journal of Magnetic Resonance Imaging. 2002;16(4):451-463 %@ 1522-2586.


