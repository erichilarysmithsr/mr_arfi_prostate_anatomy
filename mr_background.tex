\subsection{Clinical Background and Significance}
Prostate cancer (PCa) is the most common non-cutaneous malignancy among men in
the United States. Approximately 1 in every 6 men will develop PCa during their
lifetime, with the median age of diagnosis at 67. 1It is also the second
leading cause of cancer related death, with 1 in 36 men dying from the disease.
The National Cancer Institute estimates that 238,590 men will be diagnosed with
PCa in 2013 and 29,720 will die from the disease.~\cite{Howlader2011}

PCa diagnosis usually begins by screening with prostate specific antigen (PSA)
and digital rectal examination (DRE).  Definitive diagnosis is made by random
transrectal ultrasonography-guided (TRUS) biopsies, which are then used to
provide the clinician with the proper Gleason score. The combination of these
factors—as well as staging—determines the appropriate therapy and prognosis. 

PCa screening has led to earlier diagnosis of smaller tumors and more localized
disease.  However, it is well known that the sensitivity and specificity of PSA
and DRE are not optimal. In addition, DRE has a low predictive value at lower
PSA ranges, and PSA yields many false
positives.~\cite{Gosselaar2007,Gupta2013,Hricak2007} As such, a theoretical
risk of over-diagnosis and treatment of low-grade—and possibly clinically
insignificant—disease exists. Moreover, due to the random nature of biopsies,
cancer located outside the routine sampling site can be missed and the extent
of the cancer might be underestimated.~\cite{Gupta2013,Cornud1012} For example,
in a study by Mufarrij et al. 45.9--47.2\% of patients who were candidates for
active surveillance but underwent radical prostatectomy had a higher Gleason
score on final histopathology than after TRUS biopsy.~\cite{Mufarrij2010} These
inaccuracies may lead to inappropriate diagnosis, imprecise risk assessment and
potentially avoidable morbidity.

\subsection*{The Use of Magnetic Resonance Imaging in Prostate Diagnostics}
Magnetic resonance (MR) imaging has been available for use in the workup of
patients with PCa since the early 1980s but the early work on its diagnostic
accuracy is heterogeneous.7 Earlier MR techniques relied mostly on morphology
via T1 and T2- weighted imaging (T2WI). The more recent ability to include not
only anatomic but also biologic and functional dynamic parameters into MR
analysis—via diffusion-weighted imaging (DWI), dynamic contrast-enhanced (DCE)
imaging or MR spectroscopic imaging (MRSI)—is promising in the future diagnosis
and management of PCA.

Currently prostate MR focuses on a multiparametric approach, where 2 or more
imaging sequences—including anatomic and functional data—are used together to
try to arrive to a diagnosis.~\cite{Barentsz2012} As MR technology continues to evolve and
improve, its role in PCA diagnosis, staging, treatment planning and follow-up
has gained much attention.

\subsection{T2-Weighted Imaging and Prostate Anatomy}
T2WI sequences are crucial components of prostate MR imaging.  T2WI is
particularly useful in prostate analysis due to its excellent soft tissue
contrast resolution, which can be maximized by using thin sections of 3--4 mm
and a small field of view of approximately 14 cm.~\cite{Gupta2013,Bonekamp2011}
T2 sequences are the most helpful for tumor localization as they can clearly
show overall prostate morphology, internal structures and prostatic
margins.~\cite{Gupta2013}

The prostate can be divided into glandular and non-glandular components. The
glandular components include the peripheral zone (PZ) and the central gland,
which are typically easily distinguishable on T2WI. The central gland includes
the central zone, transition zone and the periurethral glandular
tissue.~\cite{Jung2012} Other anatomical markers such as the urethra,
verumontanum and ejaculatory are also often seen on T2WI. 

Approximately 70\% of the prostatic tissue is found in the peripheral zone,
which is high in water content and thus of higher signal intensity in
T2WI.~\cite{Jung2012} Seventy five percent of prostatic tumors are found in the
PZ and normally show hypointense T2 signal when compared to the higher
intensity PZ.~\cite{Hricak2007,Hegde2013} However, tumors can sometimes seem of
similar intensity as the surrounding tissue and false positives can occur
secondary to post biopsy changes/hemorrhage, hyperplasia or prostatitis, making
diagnosis more challenging.~\cite{Hegde2013}

\subsection{Functional MR Sequences}
Even though T2WI is the mainstay of prostate MR, its overall performance in
prostate cancer diagnosis is not optimal. The incorporation of two or more
functional sequences in multiparametric MR imaging (mpMRI) has been shown to
significantly improve the performance of MRI in cancer
diagnosis.~\cite{Turkbey2012} In fact, the European Society of Urogenital
Radiology’s (ESUR) prostate MR guidelines recommend at least 2 functional
imaging techniques in addition to T2WI in order to better characterize prostate
tumors.~\cite{Barentsz2012} In a study by Turkbey et al., researchers found
that mpMRI had a positive predictive value of 98\% in prostate cancer
detection.~\cite{Turkbey2012} Functional sequences include diffusion- weighted
imaging (DWI), dynamic contrast enhanced imaging (DCE-MRI) and MR spectroscopic
imaging (MRSI). 

DWI is based on the free movement of water particles in tissue and measures the
degree of motion restriction.~\cite{Koh2007} Normal prostatic tissue is very
glandular with plenty of water molecule movement. On the other hand, tumors
have high cellular density and restricted water movement, which leads to
decreased diffusion. Due to this restricted diffusion, tumors seem to be of
higher intensity on DWI.~\cite{Koh2007}

DCE-MRI provides a measure of tumor vascularity and takes advantage of
tumor-induced angiogenesis in PCa.~\cite{Noworolski2005} Increased vascularity
in PCa results in earlier and higher peak contrast enhancement and a faster
washout when compared to normal prostatic tissue. These known characteristics
of PCa are helpful when trying to localize lesions. 

Like DWI, MRSI can be helpful in lesion characterization as it provides
information on the presence of certain metabolites in
tissue.~\cite{Barentsz2012} Choline and citrate are especially useful in the
setting of PCa. Choline is critical in cell membrane synthesis and is thus
usually elevated in cells with cancerous behavior.~\cite{Gupta2013} Healthy
prostate epithelium synthesizes and secretes large quantities of citrate but
levels are decreased in PCa.~\cite{Costello1999,Kurhanewicz2002} Therefore, an
increase in the choline-to-citrate ratio on MRSI can be used as an indicator of
malignancy.~\cite{Bonekamp2011}
