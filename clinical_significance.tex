\subsection{Clinical Background and Significance}
PCa is the most common non-cutaneous malignancy among men in the United States.
Approximately 1 in every 6 men will develop PCa during their lifetime, with the
median age of diagnosis at 67 years old.~\cite{Howlader2011} PCa is also the
second leading cause of cancer-related death, with 1 in 36 men dying from the
disease.  The National Cancer Institute (NCI) estimates that 238,590 men will
be diagnosed with PCa in 2013 and 29,720 will die from the
disease.~\cite{Howlader2011}

PCa diagnosis usually begins by screening with prostate specific antigen (PSA)
and digital rectal examination (DRE).  Definitive diagnosis is made by random
transrectal ultrasound (TRUS)-guided biopsies, which are then used to provide
the clinician with the proper Gleason score. The combination of these factors,
as well as staging, determines the appropriate therapy and prognosis. 

PCa screening has led to earlier diagnosis of smaller tumors and more localized
disease; however, it is well known that the sensitivity and specificity of PSA
and DRE are not optimal. In addition, DRE has a low predictive value at lower
PSA ranges, and PSA yields many false
positives.~\cite{Gosselaar2007,Gupta2013,Hricak2007} As such, a theoretical
risk of over-diagnosis and treatment of low-grade, and possibly clinically
insignificant, disease exists.  Moreover, due to the random nature of
TRUS-guided systematic biopsies, PCa located outside the routine sampling sites
can be missed and the extent of the cancer might be
underestimated.~\cite{Gupta2013,Cornud2012} For example, in a study by Mufarrij
\etal, 45.9--47.2\% of patients who were candidates for active surveillance,
but underwent radical prostatectomy, had a higher Gleason score on final
histopathology than after TRUS biopsy.~\cite{Mufarrij2010} These inaccuracies
may lead to inappropriate diagnosis, imprecise risk assessment and potentially
avoidable morbidity.

