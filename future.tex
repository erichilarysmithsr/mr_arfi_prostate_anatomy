\subsection{Future Directions}
Future directions for our research include deeper analysis of our findings,
especially the lateral dimension over-estimation, to improve our anatomic
delineation and assess further its clinical impact.  Technological improvements
to improve anterior prostate boundary delineation in both B-mode and ARFI
imaging are also being pursued and will hopefully allow for the currently
elusive anterior PCa lesions in MR to be seen in ARFI imaging.  3D models of
the prostate and central glands will be used across ARFI and MR imaging
datasets to spatially register them to facilitate correlation of regions of PCa
suspicion between the two modalities, and to correlate with whole-mount
histology of the excised prostate specimens.  This effort will allow the
sensitivity and specificity of each imaging modality for PCa detection and
characterization to be quantified.  While ARFI imaging of the prostate in a
constantly evolving technology, the fact that one can view the different
prostatic zones on ARFI images in addition to MRI is encouraging and could lead
to targeted diagnostic biopsies and therapies with real-time imaging for men
with PCa. 

