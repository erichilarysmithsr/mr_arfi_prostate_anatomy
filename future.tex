\subsection{Future Directions}
Future directions for our research include deeper analysis of our findings,
especially the lateral dimension over-estimation, to improve our anatomic
delineation and assess further its clinical impact.  Technological improvements
to improve anterior prostate boundary delineation in both B-mode and ARFI
imaging are also being pursued for better visualization in this region.  3D
models of the prostate and central glands will be used across ARFI and MR
imaging datasets to spatially register them to facilitate correlation of
regions of PCa suspicion between the two modalities, and to correlate with
whole-mount histology of the excised prostate specimens.  
\textbf{Diffeomorphic image registration techniques utilizing these 3D capsules
will be used to accomodate the different prostate deformations associated with
each imaging modality, allowing for a more direct, voxel-to-voxel comparison of
the different imaging modalities and how they may complement each other.}

These
efforts for improved imaging performance and multi-modality image registration
will allow for the quantification of sensitivity and specificity of a combined
MR/B-mode/ARFI imaging system for PCa detection and characterization.
Additionally, we will be pursuing quantitative measurement of the pressure
being applied during imaging to help reduce prostate deformation and reduce the
impact of prostate stiffness nonlinearity in our displacement images.  While
ARFI imaging of the prostate is a constantly evolving technology, the fact that
one can view the different prostatic zones on ARFI images in addition to MRI is
encouraging and could lead to targeted diagnostic biopsies and therapies with
real-time imaging for men with PCa. 

