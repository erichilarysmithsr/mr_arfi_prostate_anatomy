\subsection{Limitations}
This study has several limitations that should be considered when interpreting
these results.  Gross pathology weight and axis measurements could both be
affected by the presence of peri-prostatic tissue that was excised during
radical prostatectomy, especially in cases where more aggressive margins may
have been necessary.  For this reason, unlike the image-to-image measurement
comparisons, all of the image metrics presented relative to pathology metrics
(Figure~\ref{fig:mr_arfi_weight}) were not characterized for absolute accuracy,
but instead, relative correlations were evaluated.  Additionally, the volumes
of the prostate from gross pathologic measurements were approximated as
ellipsoids, which also introduced error, most likely an over-estimation of
volume.  Interestingly, all pathology estimates were thought to have positive
biases, but both imaging modalities tended to overestimate volume relative to
the pathology measurements.

It should also be noted that all of the prostates in this study contained
varying amounts of PCa, BPH and atrophy, all of which can distort the zonal
anatomy, especially in the case of BPH and central gland morphology.  While
younger, healthier prostates could have been targeted, these healthy organs
would not have been excised for pathology characterization, and the zonal
anatomy of a healthy (young) prostate is expected to be different from the
prostate of a middle-age man, who is the target demographic for PCa screening
imaging and PCa characterization. 

This study did not evaluate user biases in image segmentation.  While MR zonal
anatomy delineation has some establishment in the clinical literature, this
study is the first attempt to define the criteria for ARFI imaging zonal
anatomy characteristics, and it is expected that such delineations will
continue to be refined as we acquire more cases and continue to compare with
MRI and pathology data.  Given this limitation, no attempts were made to
further quantify reader-to-reader variability in this work.
