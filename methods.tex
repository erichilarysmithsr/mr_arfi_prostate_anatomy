\subsection{Study Inclusion Criteria \& MR Imaging}
Prostates removed by radical prostatectomy were used in this IRB-approved (Duke
IRB\# Pro00006458), HIPAA-compliant study from men ranging in age from XX--XX
diagnosed with biopsy-proven prostate cancer destined for surgical removal.
Between DATES, a total of XX patients were recruited and enrolled in this
study. Inclusion criteria were undergoing complete pelvic MRI with endorectal
coil for detection of prostate cancer, including multiplanar T2-weighted
anatomic imaging, diffusion-weighted imaging (DWI), and dynamic contrast
enhanced MRI (DCE-MRI) as well as radical prostatectomy and whole mount
histology. Patients with previous treatments of prostate cancer or benign
prostatic hyperplasia (BPH), or anatomic anomalies of the rectum, were
excluded.  All patients enrolled in this study provided written informed
consent. Our final cohort included XX subjects. Table XX has a summary of the
patient demographics for the cases shown in this manuscript.  

All imaging was performed on one of two 3.0 Tesla MR scanners (General Electric
HDx, GE Healthcare, Waukesha, WI;  Siemens Skyra, Siemens Healthcare, Erlangan,
Germany) using a single channel Medrad eCoil endorectal coil (Medrad,
Indianola, PA) as well as multichannel surface coils.  Imaging sequences
included thin-section (3 mm section thickness) fast spin echo T2-weighted
images in the coronal, axial and sagittal planes.  Diffusion weighted images
were obtained using multiple b-values and calculation of ADC maps was also
performed.  Dynamic contrast enhanced MR sequences were obtained after
administration of a weight-based dose of extracellular MR contrast agent with
4-5 second temporal resolution for 5-6 minutes. (If we need it, can put in
table with full MR parameters) Prostates were radically removed using a da
Vinci Surgical System (INSERT COMPANY INFORMATION).  After exision, the
prostates were formalin fixed for at least 24 hours without being cut, and then
processed for whole mount histology.

\subsection{ARFI Imaging}

\subsection{Image Zonal Anatomy Segmentation and 3D Model Rendering}
