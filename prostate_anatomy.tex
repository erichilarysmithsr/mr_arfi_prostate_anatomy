\subsection{Prostate Anatomy}

The prostate gland sits below the urinary bladder and surrounds the urethra.
Its superior borders include the bladder and seminal vesicles and the
urogenital diaphragm delineates its inferior boundary. The gland is bordered
anteriorly by the pubic symphysis and posteriorly by the rectum.  The prostate
is separated from the rectum by 2--3 mm fascial layer,~\cite{Jung2012} and can
be easily palpated on rectal examination. 

The gland can be divided from superior to inferior into the base, midgland and
apex. The urethra enters the prostate proximally at the base and extends to the
midgland at which point the ejaculatory ducts open into the urethra at the
verumontanum.~\cite{Jung2012} The urethra then continues past the apex and
travels through the penis. The prostate can be divided into glandular and
non-glandular components.  The glandular components include the transitional
zone, central zone and peripheral zone. Each zone contains approximately 5\%,
20\% and 70--80\% of glandular tissue, respectively.~\cite{Bonekamp2011} The
non-glandular components include the anterior fibromuscular stroma and the
urethra. 

Although not a true capsule, an outer band of fibromuscular tissue surrounds
the prostate.~\cite{Bonekamp2011} This ``capsule'' is important when assessing
the extraprostatic extension of cancer as tumor can spread by disrupting this
tissue. Two neurovascular bundles course posterior and lateral to the prostate,
which can also be invaded by malignant cells.  
