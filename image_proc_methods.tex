\subsection{Image Zonal Anatomy Segmentation and 3D Model Rendering}
Axial MR T2WI images were manually segmented using the smooth polygon tool in
ITK-SNAP~\cite{Yushkevich2006}, using unique labels for the PZ, CG and AFS. The
gland was segmented from base to apex.  The base was identified below the
bladder, and subsequent images were segmented until the last slice with visible
prostatic tissue was identified caudally. The CG, PZ and AFS were segmented
independently according to their well-established anatomical characteristics on
T2WI.~\cite{Verma2011,Jung2012,Poon1985,Hricak2007,Bonekamp2011} The PZ was
identified by its homogenous high signal intensity on T2WI, which is usually
similar to that of the nearby periprostatic fat. The CG was visualized and
delineated based on its heterogeneous and lower signal intensity as well as its
location (Figure~\ref{fig:mr_anatomy}). Although not readily visible on every
case, the AFS was identified by its low T2 signal intensity and its location
anterior to the central gland. 

ARFI images were segmented using a similar procedure to the MR images.  B-mode
images were used to segment the prostate capsules since contrast between the
peripheral zone and the periprostatic fat can be low, and poor displacement SNR
can be present along the anterior aspect of the prostate in ARFI images, making
it difficult to delineate that boundary.  The central gland was segmented as a
stiffer region, relative to the peripheral zone, in the center of the prostate.

\emph{CO-AUTHOR QUESTION: CAN WE EASILY DELINEATE OUR SEGMENTATION CRITERIA?}

Segmented image stacks were imported into 3D Slicer (v4.3.0) and 3D
models were rendered using the following parameters (Table~\ref{tab:3dslicer}):

\begin{table}[h!]
\centering
\caption{3D model volume rendering parameters}
\begin{tabular}{ll}
{\bf Parameter} & {\bf Value} \\ \hline
Decimation & 0.1 \\
Smoothing Algorithm & Laplacian \\
Smoothing  & 70.0 \\
Joint Smoothing & Enabled \\
\end{tabular}
\end{table}

The 3D slicer models were used to render volume estimates of the PZ and CG,
with the sum of PZ and CG representing the total prostate gland volume.  AFS
volume estimates in select MR cases were included in the total prostate gland
volume estimates.  Orthogonal tri-axial measurements in the lateral-to-lateral,
axex-to-base, and anterior-to-posterior dimensions were made in 3D slicer.
