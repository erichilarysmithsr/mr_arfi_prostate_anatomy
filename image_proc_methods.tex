\subsection{Image Zonal Anatomy Segmentation and 3D Model Rendering}
Axial T2W MR slices were manually segmented using the polygon tool on ITK SNAP
using separate labels for the peripheral zone (PZ), central gland (CG) and
anterior fibromuscular stroma (AFS). The gland was segmented from base to apex.
The base was identified below the bladder and subsequent images were segmented
until the last slice with visible prostatic tissue was identified caudally. The
CG, PZ and AFS were segmented independently according to their well-established
anatomical characteristics on
T2WI.~\cite{Verma2011,Jung2012,Poon1985,Hricak2007,Bonekamp2011} The PZ was
identified by its homogenous high signal intensity on T2WI, which is usually
similar to that of the nearby periprostatic fat. The CG was visualized and
delineated based on its heterogeneous and lower signal intensity as well as its
location. Although not readily visible on every case, the AFS was identified by
its low T2 signal intensity and its location anterior to the central gland. 
