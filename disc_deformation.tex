\textbf{We did not expect perfect agreement between MR and ultrasound imaging
volumes or tri-axial dimensions since each modality applies a different amount
of external compression to the rectal wall, which results in different degrees
of prostate deformation The endorectal coil used in MRI places pressure on the
prostate that is dependent on the patient's anatomy and overall prostate size,
while the pressure on the prostate during transrectal ultrasound imaging is
modulated by the urologist when the transducer rotation apparatus is locked
into position prior to imaging.  The differential prostate deformation during
transrectal ultrasound imaging can be easily visualized during the transducer
alignment process.  It was our goal to apply as little pressure while still
maintaining uniform contact throughout the ultrasound image acquisition. This
is critical since poor acoustic coupling to the prostate through the rectal
wall can greatly diminish ARFI image displacement SNR.}
