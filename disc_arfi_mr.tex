\subsection{ARFI and MR T2WI Agreement}
This study demonstrated that ARFI imaging delineates zonal anatomy of the
prostate well when compared to MR T2WI, with comparable estimates of total and
central gland volume (Figure~\ref{fig:mr_arfi_volumes}).  Establishing accurate
delineation of the prostate's zonal anatomy is important since PCa has strong
correlations with location in the prostate, specifically in the peripheral
zone, and accurate capsule delineation will facilitate future efforts to
register 3D imaging datasets between ARFI and MR imaging.  

This was the first quantitative comparison between ARFI imaging and a newly
established clinical imaging modality to evaluate the prostate for PCa
detection and characterization, and we have drawn some useful conclusions in
how our zonal anatomy boundaries are determined during image segmentation.
Poor contrast between the PZ and peri-prostatic fat in B-mode images can make
delineation of the prostate boundary in the lateral dimension challenging,
emphasizing the need to use the orthogonal imaging plane views to help endure
organ continuity in all three dimensions.  Our study demonstrated an
overestimation of lateral boundaries of the prostate in our ultrasound datasets
compared to MR T2WI, leading to an overall overestimation of prostate total
gland volume.  We did not expect perfect agreement between MR and ARFI imaging
in this dimension since each modality applies a different amount of external
compression to the rectal wall that can deform the prostate different amounts.
Presure on the prostate during ARFI imaging is modulated by the urologist when
he locks the transducer rotation apparatus in place prior to imaging, and
deformation of the prostate can be easily visualized during this alignment
process.  It was our goal to apply as little pressure to the prostate during
imaging, while still maintaining uniform contact throughout all imaging angles,
which is critical since poor acoustic coupling to the prostate through the
rectal wall can greatly diminish ARFI image displacement SNR.

The anterior aspect of the prostate can also be challenging to delineate,
especially in large prostates, where SNR decreases with increasing depth away
from our rectal wall imaging surface and MR images suffer from gland
heterogeneity in this region~\cite{Gupta2013}, though we had good agreement
between ARFI and MR T2W images in the anterior-to-posterior dimension
(Figure~\ref{fig:mr_arfi_path_axes}(b), Table~\ref{tab:mr_arfi_axes_error}).

ARFI imaging consistently underestimated the extent of the prostate capsule and
CG along the apex-to-base axis relative to MR T2WI
(Figure~\ref{fig:mr_arfi_path_axes}(c), Table~\ref{tab:mr_arfi_axes_error}).
This was the dimension with the worst spatial resolution in MR imaging, with a
typical slice thickness in MR T2WI being $\sim$3 mm for these studies.  While
higher-resolution MR images could be obtained and analyzed for more precise
comparisons in this imaging dimension, such data was not available for this
study.  Additionally, it should also be noted that the precision in ARFI
imaging for this study is also constrained by the decimation used to reduce the
number of imaging planes manually segmented, introducing a lower-bound on the
model precision of, again, $\sim$3 mm.  While less decimation could improve the
precion of the prostate and GG axis measurements, this would have introduced
sigificant overhead in the manual segmentation process, with diminishing
returns given the spatial smoothing of the modeling algorithms that were
utilized (Table~\ref{tab:3dslicer}).
