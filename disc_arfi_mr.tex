\subsection{ARFI and MR T2WI Agreement}
This study demonstrated that ARFI imaging delineates zonal anatomy of the
prostate well when compared to MR T2WI with only slight overestimation of ARFI
total and central gland volume (Figure~\ref{fig:mr_arfi_volumes}).
Establishing accurate delineation of the prostate's zonal anatomy is important
because PCa has strong correlations with location in the prostate, specifically
in the peripheral zone, and accurate capsule delineation will facilitate
future efforts to register 3D imaging datasets between ARFI and MR imaging. 

We did not expect perfect agreement between MR and ARFI imaging volume or
tri-axial dimensions since each modality applies a different amount of external
compression to the rectal wall, which then deforms the prostate.  Pressure on
the prostate during ARFI imaging is modulated by the urologist when he locks
the transducer rotation apparatus in place prior to imaging, and deformation of
the prostate can be easily visualized during this alignment process.  It was
our goal to apply as little pressure while still maintaining uniform contact
throughout the image acquisition. This is critical since poor acoustic coupling
to the prostate through the rectal wall can greatly diminish ARFI image
displacement SNR.

The lateral boundaries of the prostate in the U/S dataset were overestimated
relative to MR T2WI, increasing ARFI total volume relative to MR. The lateral
overestimation was a result of poor contrast between the PZ and peri-prostatic
fat in the B-mode images, and image decimation along the lateral dimension in
both ARFI and B-mode images. Without image decimation, the elevation and axial
orthogonal planes can be used to correct errors introduced by the poor contrast
between the PZ and peri-prostatic fat. However, with image decimation,
segmentation changes in the elevation and axial orthogonal planes were limited
to the lower bound on the model precision of, again, $\sim$3 mm.  While reduced
decimation could improve the precision of the lateral measurements, this would
have introduced significant overhead in the manual segmentation process, with
diminishing returns given the spatial smoothing of the modeling algorithms that
were utilized. 

The anterior aspect of the prostate can also be challenging to delineate,
especially in large prostates, where SNR decreases with increasing depth away
from our rectal wall imaging surface and MR images suffer from gland
heterogeneity in this region~\cite{Gupta2013}, though we had good agreement
between ARFI and MR T2W images in the anterior-to-posterior dimension
(Figure~\ref{fig:mr_arfi_path_axes}(b), Table~\ref{tab:mr_arfi_axes_error}).

ARFI imaging consistently underestimated the extent of the prostate capsule and
central gland apex-to-base dimension relative to MR T2WI
((Figure~\ref{fig:mr_arfi_path_axes}(c), Table~\ref{tab:mr_arfi_axes_error}).
In fact, it is likely the degree of ARFI estimation relative to the actual
apex-to-base prostate length is even greater than suggested by the MR
comparison as this was the dimension with the worst spatial resolution in MR
imaging, with a typical MR T2WI slice thickness of $\sim$3 mm for these
studies.  While higher-resolution MR images could be obtained and analyzed for
more precise comparisons in this imaging dimension, such data were not available
for this study. 
