\section{Introduction}\label{sect:intro}
Prostate cancer (PCa) is the most common non-cutaneous malignancy among men in
the United States and the second leading cause of cancer related
death.~\cite{Howlader2011} Screening with prostate specific antigen (PSA) and
digital rectal examination (DRE) has led to earlier PCa detection, but
performance of these measures is not optimal, leading to imprecise risk
assessment.  Additionally, the random nature of transrectal ultrasound
(TRUS)-guided biopsies can miss or underestimate the burden of
cancer.~\cite{Gupta2013} Earlier disease diagnosis leads to challenges in
deciding optimal management strategies for patients presenting with less
disease burden, whereas missed tumors on TRUS-guided biopsies can result in
inappropriate diagnoses. 

The use of non-invasive prostate imaging could lead to improved PCa diagnosis,
risk-stratification, and management.  Magnetic resonance imaging (MRI) has been
available for use in the workup of patients with PCa since the early 1980s, but
early studies on its diagnostic accuracy were heterogeneous.  The more
recent ability to include functional parameters in PCa MRI analysis has yielded
promising results.~\cite{Gupta2013,Hricak2007} Among the MRI modalities
currently used in the study of PCa, it is well established that T2-Weighted
Imaging (T2WI) offers the best assessment of prostate anatomy due to its
ability to delineate prostatic margins, distinguish internal structures and
differentiate among the glandular zones. 

Acoustic Radiation Force Impulse (ARFI) imaging is an ultrasound-based modality
that evaluates the mechanical properties of tissues.~\cite{Nightingale2002b}
ARFI imaging has the potential to aid in PCa diagnosis and management by
evaluating the structural composition of prostate zones and tumors based on
their stiffness contrast.  Zhai \etal were able to visualize prostatic anatomy
by utilizing ARFI imaging in freshly-excised prostates. In a second study, Zhai
\etal demonstrated the feasibility of ARFI prostate imaging
\invivo~\cite{Zhai2012}; however, to the authors’ best knowledge, there have
been no studies to date that compare \invivo ARFI prostate imaging to other
imaging modalities.~\cite{Zhai2010} 

The goal of this study is to evaluate the ability of ARFI imaging to delineate
prostate zonal anatomy, specifically central gland and total prostate gland
\invivo as compared to endorectal MR T2WI.  Section~\ref{sect:background}
provides an overview of MR and ARFI imaging in the prostate and an overall
clinical motivation for prostate imaging.  Section~\ref{sect:methods} describes
the methods used to experimentally-acquire our imaging data, process of gross
pathological specimens post radical prostatectomy, and image process our
datasets.  The results of our analysis, including gross pathology and imaging
prostate axis and volume estimates, are presented in
Section~\ref{sect:results}, along with a statistical analysis of the bias and
variability associated with each imaging modalities measurements, all of which
is discussed in Section~\ref{sect:discussion}.
