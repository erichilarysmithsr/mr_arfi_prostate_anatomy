\section{Introduction}\label{sect:intro}
Prostate cancer (PCa) is the most common non-cutaneous malignancy among men in
the United States and the second leading cause of cancer related
death.~\cite{Howlader2011} Approximately 1 in every 6 men will develop PCa
during their lifetime, with the median age of diagnosis at 67 years
old.~\cite{Howlader2011} PCa is also the second leading cause of cancer-related
death, with 1 in 36 men dying from the disease.  The National Cancer Institute
(NCI) estimates that 238,590 men will be diagnosed with PCa in 2013 and 29,720
will die from the disease.~\cite{Howlader2011}

PCa diagnosis usually begins by screening with prostate specific antigen (PSA)
and digital rectal examination (DRE).  More definitive diagnosis can be made by
random transrectal ultrasound (TRUS)-guided biopsies, which are then used to
provide the clinician with the proper Gleason score. The combination of these
factors, as well as staging, determines the appropriate therapy and prognosis.

PCa screening has led to earlier diagnosis of smaller tumors and more localized
disease; however, it is well known that the sensitivity and specificity of PSA
and DRE are not optimal. In addition, DRE has a low predictive value at lower
PSA ranges, and PSA yields many false
positives.~\cite{Gosselaar2007,Gupta2013,Hricak2007} As such, a theoretical
risk of over-diagnosis and treatment of low-grade, and possibly clinically
insignificant, disease exists.  Moreover, due to the random nature of
TRUS-guided systematic biopsies, PCa located outside the routine sampling sites
can be missed and the extent of the cancer might be
underestimated.~\cite{Gupta2013,Cornud2012} For example, in a study by Mufarrij
\etal, 45.9--47.2\% of patients who were candidates for active surveillance,
but underwent radical prostatectomy, had a higher Gleason score on final
histopathology than after TRUS biopsy.~\cite{Mufarrij2010} These inaccuracies
may lead to inappropriate diagnosis, imprecise risk assessment and potentially
avoidable morbidity.

The use of non-invasive prostate imaging could lead to improved PCa diagnosis,
risk-stratification, and management.  Magnetic resonance imaging (MRI) has been
available for use in the workup of patients with PCa since the early 1980s, but
early studies on its diagnostic accuracy were heterogeneous.  The more
recent ability to include functional parameters in PCa MRI analysis has yielded
promising results.~\cite{Gupta2013,Hricak2007} Among the MRI sequences
currently used in the study of PCa, it is well established that T2-Weighted
Imaging (T2WI) offers the best assessment of prostate anatomy based on its
ability to delineate prostatic margins, distinguish internal structures and
differentiate among the glandular zones. 

Acoustic Radiation Force Impulse (ARFI) imaging is an ultrasound-based modality
that evaluates the mechanical properties of tissues.~\cite{Nightingale2002b}
ARFI imaging has the potential to aid in PCa diagnosis and management by
evaluating the structural composition of prostate zones and tumors based on
their stiffness contrast.  Zhai \etal~were able to visualize prostatic anatomy
by utilizing ARFI imaging in freshly-excised prostates. In a second study, Zhai
\etal~demonstrated the feasibility of ARFI prostate imaging
\invivo~\cite{Zhai2012}; however, to the authors’ best knowledge, there have
been no studies to date that compare \invivo ARFI prostate imaging to other
imaging modalities.~\cite{Zhai2010} 

The goal of this study is to evaluate the ability of ARFI imaging to delineate
prostate zonal anatomy, specifically central gland and total prostate gland
\invivo as compared to T2WI MR at 3T using an endorectal coil.
Section~\ref{sect:background} provides an overview of MR and ARFI imaging in
the prostate and an overall clinical motivation for prostate imaging.
Section~\ref{sect:methods} describes the methods used to experimentally-acquire
our imaging data, process the gross pathological specimens post radical
prostatectomy, and image process our datasets.  The results of our analysis,
including gross pathology and imaging prostate axis and volume estimates, are
presented in Section~\ref{sect:results}, along with a statistical analysis of
the bias and variability associated with each imaging modalities measurements,
all of which is discussed in Section~\ref{sect:discussion}.
