\subsection{Acoustic Radiation Force Impulse (ARFI) Imaging}
ARFI imaging is an ultrasound technique that evaluates the mechanical
properties of tissues. It generates short-duration acoustic radiation forces
that result in tissue displacement. The response can then be measured for
tissue characterization, with displacement being inversely proportional to
stiffness.~\cite{Nightingale2002} This characterization is of particular
importance in prostate evaluation due to the gland’s complex and heterogeneous
composition.  Previous studies have shown that the central zone and PCA can be
up to 3 times stiffer than PZ tissue.~\cite{Zhai2010a,Zhai2012,Zhai2010b} This
finding can be explained by the PZ’s higher water content and is analogous to
the higher signal intensity of the PZ on T2WI.

The potential of ARFI imaging to distinguish normal prostatic anatomy from
pathologic tissue is promising in the future of PCa diagnosis and image guided
therapies. B-mode US, which is currently used for real-time guidance in
TRUS-guided biopsies, does not have the ability to clearly visualize PCa or to
differentiate it from other normal structures or disease processes such as
benign prostatic hyperplasia (BPH) or prostatitis.~\cite{Zhai2010a} This has
resulted in a systematic yet random approach to prostate core biopsies, with
6-12 cores routinely sampled from different anatomical regions.~\cite{Loch2004}
The ability to clearly distinguish PCa from other tissues could lead to
targeted biopsies with increased sensitivity and resultant decreased morbidity.
It could also aid in the development of targeted therapies for those with low
burden of disease. 
