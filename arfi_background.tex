\subsection{Acoustic Radiation Force Impulse (ARFI) Imaging}
Acoustic radiation force-based elasticity imaging has recently been developed
for investigation of mechanical properties of human soft tissues, including
liver, arteries, heart, prostate, breast, and cervix~\cite{Bouchard2009,Dumont2009,Trahey2004,Dahl2009,proc_congdon04,Palmeri2011,Zhai2010a,hsu07,Palmeri2011a,Palmeri2013}.
The acoustic radiation force $\vec{F}$ generated in soft tissues by
focused ultrasound can be described by~\cite{nyborg65,torr84}:
\begin{equation}
\vec{F} = \frac{2 \alpha \vec{I}}{c},
\label{eqn:radforce}
\end{equation}
where $\alpha$ is the acoustic attenuation coefficient of the tissue, $c$ is
the tissue's sound speed, and $\vec{I}$ is the acoustic intensity at a given
point in space.  The acoustic radiation force is generated by a transfer of
momentum from the propagating acoustic wave to the propagation medium through
attenuation mechanisms, such as absorption and scattering of the ultrasonic
wave.  In these studies, the acoustic attenuation and sound speed of the
prostate tissue were assumed to be constant and not vary as a function of the
zonal anatomy, meaning that all displacement differences seen in the prostate
would be assumed to be stiffness-related after normalizing for depth-dependent
focal gain.
