\section*{Abstract}
Prostate cancer (PCa) is the most common non-cutaneous malignancy among men in
the United States and the second leading cause of cancer-related death.
Two imaging modalities, Multi-parametric Magnetic Resonance Imaging (mpMRI) and
Acoustic Radiation Force Imaging have the potential to aid in PCa
diagnosis and management by evaluating the structural composition of prostate
zones and tumors based on water content and stiffness respectively.  In this study, 
MR and ARFI imaging datasets were compared to one another and against gross pathology in order to evaluate their
ability to delineate central gland (CG) and prostate capsule anatomy. 3D slicer, an open source image processing program,  was used for MR and ARFI manual
 for manual segmentation, 3d modeling, and measurement of zonal anatomy tri-axial dimensions and volumes.   
 Both imaging modalities showed moderate correlations between
modeled organ volume and gross pathologic weights.  In the intra-modality comparison,
ARFI and MR total prostate gland volumes were well-correlated (R$^2$ = \MRarfiVolTotalRsq), but
ARFI yielded prostate volumes that were, on average, larger
(\MRarfiVolTotalMeanDiff\% $\pm$ \MRarfiVolTotalStdDiff\%) than MR. This was
primarily due to total gland over-estimation along the lateral dimension.
(\ARFImrTotalLatLatMeanPct $\pm$ \ARFImrTotalLatLatStdPct\%). Differences between MR and
ARFI in the other dimensions were less significant (\ARFImrTotalAntPostMeanPct $\pm$
\ARFImrTotalAntPostStdPct\% and \ARFImrTotalApexBaseMeanPct $\pm$
\ARFImrTotalApexBaseStdPct\%). ARFI and MR CG volumes were also
well-correlated (R$^2$ = \MRarfiVolCentralRsq).  CG volume differences were
attributed to MR apex-to-base axis underestimation 
(\ARFImrCentralApexBaseMeanPct $\pm$ \ARFImrCentralApexBaseStdPct\%), and
ARFI lateral dimension over-estimation  (\ARFImrCentralLatLatMeanPct
$\pm$ \ARFImrCentralLatLatStdPct\%).  Overall, ARFI prostate imaging yielded prostate
volumes and dimensions that were correlated with MR T2WI estimates, with biases
in the lateral dimension. ARFI imaging is a promising low-cost, real-time imaging
modality that can compliment mpMRI for diagnosis, treatment planning and
management of PCa.

