\section*{Abstract}
Prostate cancer (PCa) is the most common non-cutaneous malignancy among men in
the United States and the second leading cause of cancer related death.
Non-invasive imaging could lead to improved diagnosis, risk-stratification, and
PCa management.  Magnetic resonance imaging (MRI) has been available for use in
the workup of patients with PCa since the early 1980s, and recent advances with
functional parameters has greatly improved its clinical diagnostic utility.
Acoustic Radiation Force Impulse (ARFI) imaging is an ultrasound-based modality
that evaluates the mechanical properties of soft tissues. ARFI imaging has the
potential to aid in PCa diagnosis and management by evaluating the structural
composition of prostate zones and tumors based on their stiffness.  In this
study, MR and ARFI imaging datasets were compared to one another and with gross
pathology measurements made immediately post radical prostatectomy.  Imaging
datasets were manually segmented to delineate the central gland and prostate
capsule, and 3D models were rendered to evaluate zonal anatomy dimensions and
volumes.  Both imaging modalities showed moderate correlations (0.39 $<$ R$^2 <
$ 0.74) between estimated organ volume and gross pathologic weights.  ARFI and MR
total prostate gland volumes were well-correlated (R$^2$ = 0.63), but ARFI
images yielded prostate volumes that were, on average, larger (6.1\% $\pm$ 25\%)
than MR images, primarily due to over-estimation of the anterior-to-posterior
dimension of the prostate total gland (17.0 $\pm$ 12.1\%), while over-estimates
of the other dimensions were less significant contributors (8.1 $\pm$ 18.4\%
and 0.58 $\pm$ 12.9\%).  The central zone volumes of ARFI and MR images were
also moderately correlated (R$^2$ = 0.38), with minimal volume bias between the
imaging modalities, but significant variability case-to-case (-5.0 $\pm$
39.5\%).  Central zone volume differences were, again, strongly attributed to
over-estimation of the anterior-to-posterior axis (14.8 $\pm$ 23.1\%), with a
significant underestimation of the apex-to-base dimension (-10.8 $\pm$ 22.3\%)
and no mean bias in the lateral-to-lateral measurements (0.006 $\pm$ 17.2\%).
Strong variability in central gland volumes is believed to be related to the
extent of benign prostatic hyperplasia (BPH) for select cases.  Overall, ARFI
imaging of the prostate yielded prostate volumes and dimensions that were
correlated with MR T2WI estimates, with biases in the anterior-to-posterior
dimension, most likely related to poor displacement SNR in the anterior region
of the prostate from greater distance from the rectal wall imaging surface.
ARFI imaging is a promising low-cost, real-time imaging modality that can
compliment MR imaging for diagnosis, treatment planning and management of PCa.
