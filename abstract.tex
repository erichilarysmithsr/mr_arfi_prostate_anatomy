\section*{Abstract}
I need a few sentences from the intro here as a lead in...

MR and ARFI imaging datasets were compared with gross pathology measurements
most immediately post radical prostatectomy.  Both imaging modalities showed
moderate correlations (XX $<$ R$^2 < $ XX) between estimated organ volume and
gross pathologic weights and estimated volumes from tri-axial measurements.
ARFI images, on average, over-estimated prostate volumes by XX\% $\pm$ XX\%,
primarily due to over-estimation of the lateral (right-left) dimension of the
prostate.  MR images also over-estimated the prostate volume by XX\% $\pm$
XX\%, again, with a primary over-estimation of the lateral dimension.  MR and
ARFI imaging yielded different estimates of organ eccentricity as
characterized by ratio of the tri-axial measurements, with both imaging
modalities estimating greater degrees of eccentricity, dominated, again, by
lateral axis over-estimation.  Central gland volumes of the prostate were
well-visualized in both imaging modalities INCLUDE MORE QUANTITATIVE DETAILS
HERE.  ARFI imaging of the prostate can accurately delineate the central gland
of the prostate and the boundaries of the capsule, though care must be taken in
delineating the lateral edges of the prostate in the posterior peripheral zone.
