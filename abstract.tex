\section*{Abstract}
Prostate cancer (PCa) is the most common non-cutaneous malignancy among men in
the United States and the second-leading cause of cancer-related death.
Multi-parametric Magnetic Resonance Imaging (mpMRI) has gained recent
popularity to characterize PCa.  ARFI imaging has the potential to aid PCa
diagnosis and management by using tissue stiffness to evaluate prostate zonal
anatomy and lesions.  MR and \textbf{B-mode/}ARFI \invivo imaging datasets were
compared to one another and with gross pathology measurements made immediately
after radical prostatectomy.  Images were manually segmented in 3D Slicer to
delineate the central gland (CG) and prostate capsule, and 3D models were
rendered to evaluate zonal anatomy dimensions and volumes.  Both imaging
modalities showed good correlation between estimated organ volume and gross
pathologic weights.  Ultrasound and MR total prostate volumes were
well-correlated (R$^2$ = \MRarfiVolTotalRsq), but \textbf{B-mode} images
yielded prostate volumes that were larger (\MRarfiVolTotalMeanDiff\% $\pm$
\MRarfiVolTotalStdDiff\%) than MR images, due to overestimation of the lateral
dimension (\ARFImrTotalLatLatMeanPct $\pm$ \ARFImrTotalLatLatStdPct\%), with
less significant differences in the other dimensions
(\ARFImrTotalAntPostMeanPct $\pm$ \ARFImrTotalAntPostStdPct\%,
anterior-to-posterior, and \ARFImrTotalApexBaseMeanPct $\pm$
\ARFImrTotalApexBaseStdPct\%, apex-to-base).  ARFI and MR CG volumes were also
well-correlated (R$^2$ = \MRarfiVolCentralRsq).  CG volume differences were
attributed to ARFI underestimation of the apex-to-base axis
(\ARFImrCentralApexBaseMeanPct $\pm$ \ARFImrCentralApexBaseStdPct\%), and ARFI
overestimation of the lateral dimension (\ARFImrCentralLatLatMeanPct $\pm$
\ARFImrCentralLatLatStdPct\%).  B-mode/ARFI imaging yielded prostate volumes
and dimensions that were well-correlated with MR T2WI estimates, with biases in
the lateral dimension due to poor contrast caused by extraprostatic fat.
\textbf{B-mode combined with} ARFI imaging is a promising low-cost, portable,
real-time modality that can complement mpMRI for PCa diagnosis, treatment
planning and management.
