\section*{Abstract}
Prostate cancer (PCa) is the most common non-cutaneous malignancy among men in
the United States and the second leading cause of cancer-related death.
Multi-parametric Magnetic Resonance Imaging (mpMRI) has gained recent
popularity to characterize PCa.  ARFI imaging has the potential to aid in PCa
diagnosis and management by evaluating the structural composition of prostate
zones and tumors based on their stiffness.  In this study, MR and ARFI imaging
datasets were compared to one another and with gross pathology measurements
made immediately post radical prostatectomy.  Images were manually segmented to
delineate the central gland (CG) and prostate capsule, and 3D models were
rendered to evaluate zonal anatomy dimensions and volumes.  Both imaging
modalities showed moderate correlations between estimated organ volume and
gross pathologic weights.  ARFI and MR total prostate gland volumes were
well-correlated (R$^2$ = 0.63), but ARFI images yielded prostate volumes that
were, on average, larger (6.1\% $\pm$ 25\%) than MR images, primarily due to
over-estimation of the lateral dimension of the total gland (13.5 $\pm$
11.0\%), while over-estimates of the other dimensions were less significant
(5.7 $\pm$ 20.6\% and -2.4 $\pm$ 17.0\%).  The ARFI and MR CG volumes were also
moderately correlated (R$^2$ = 0.38).  CG volume differences were attributed to
under-estimation of the anterior-to-posterior axis (-8.4 $\pm$ 24.0\%), and
over-estimation bias in the lateral dimension (11.5 $\pm$ 22.5\%).  ARFI
prostate imaging yielded prostate volumes and dimensions that were correlated
with MR T2WI estimates, with biases in the AP dimension, most likely related to
poor displacement SNR in the anterior region of the prostate from greater
distance from the rectal wall imaging surface.  Lateral dimension contrast was
also challenging due to peri-prostatic fat.  ARFI imaging is a promising
low-cost, real-time imaging modality that can compliment MR imaging for
diagnosis, treatment planning and management of PCa.
