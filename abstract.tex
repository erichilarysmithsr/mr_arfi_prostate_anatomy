\section*{Abstract}
Prostate cancer (PCa) is the most common non-cutaneous malignancy among men in
the United States and the second leading cause of cancer related death.  The
use of non-invasive imaging in the evaluation of PCa could lead to improved
diagnosis, risk-stratification, and management.  Magnetic resonance imaging
(MRI) has been available for use in the workup of patients with PCa since the
early 1980s, and recent advances with functional parameters has greatly
improved its clinical diagnostic utility.  Acoustic radiation force impulse
(ARFI) imaging is an ultrasound-based modality that evaluates the mechanical
properties of tissues. ARFI imaging has the potential to aid in PCa diagnosis
and management by evaluating the structural composition of prostate zones and
tumors base on their stiffness.  In this study, MR and ARFI imaging datasets
were compared with gross pathology measurements immediately post radical
prostatectomy.  Both imaging modalities showed moderate correlations (0.39 $<$
R$^2 < $ 0.74) between estimated organ volume and gross pathologic weights and
estimated volumes from tri-axial measurements.  ARFI images, on average,
over-estimated prostate volumes by 36\% $\pm$ 28\% compared to MR images,
primarily due to over-estimation of the lateral (right-left) dimension of the
prostate.  The central zone volumes of the prostate agreed to within 2.1 $\pm$
39.1\% between ARFI and MR imaging.  MR and ARFI imaging yielded different
estimates of organ eccentricity as characterized by ratio of the tri-axial
measurements, with both imaging modalities estimating greater degrees of
eccentricity, dominated, again, by lateral axis over-estimation.  ARFI imaging
of the prostate can accurately delineate the central gland of the prostate and
the boundaries of the capsule, though care must be taken in delineating the
lateral edges of the prostate in the posterior peripheral zone.
