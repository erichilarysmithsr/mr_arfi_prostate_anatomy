\section*{Abstract}
Prostate cancer (PCa) is the most common non-cutaneous malignancy among men in
the United States and the second leading cause of cancer-related death.
Two imaging modalities, Multi-parametric Magnetic Resonance Imaging (mpMRI) and
Acoustic Radiation Force Imaging have the potential to aid in PCa
diagnosis and management by evaluating the structural composition of prostate
zones and tumors based on water content and stiffness respectively.  In this study, 
MR and ARFI imaging datasets were compared to one another and against gross pathology.  
MR and ARFI images were manually segmented in 3D Slicer to delineate the central gland (CG) and prostate capsule. 3D
models were then rendered in 3D Slicer to evaluate zonal anatomy tri-axial dimensions and
volumes.  Both imaging modalities showed moderate correlations between
estimated organ volume and gross pathologic weights.  In terms of intra-modality comparison,
ARFI and MR total prostate gland volumes were well-correlated (R$^2$ = \MRarfiVolTotalRsq), but
ARFI yielded prostate volumes that were, on average, larger
(\MRarfiVolTotalMeanDiff\% $\pm$ \MRarfiVolTotalStdDiff\%) than MR. This was
primarily due to total gland over-estimation along the lateral dimension.
(\ARFImrTotalLatLatMeanPct $\pm$ \ARFImrTotalLatLatStdPct\%). Differences between MR and
ARFI in the other dimensions were less significant (\ARFImrTotalAntPostMeanPct $\pm$
\ARFImrTotalAntPostStdPct\% and \ARFImrTotalApexBaseMeanPct $\pm$
\ARFImrTotalApexBaseStdPct\%). ARFI and MR CG volumes were also
well-correlated (R$^2$ = \MRarfiVolCentralRsq).  CG volume differences were
attributed to apex-to-base axis underestimation by MR
(\ARFImrCentralApexBaseMeanPct $\pm$ \ARFImrCentralApexBaseStdPct\%), and
lateral dimension over-estimation by ARFI (\ARFImrCentralLatLatMeanPct
$\pm$ \ARFImrCentralLatLatStdPct\%).  Overall, ARFI prostate imaging yielded prostate
volumes and dimensions that were correlated with MR T2WI estimates, with biases
in the lateral dimension, most likely related to poor displacement SNR in the
anterior region of the prostate from greater distance from the rectal wall
imaging surface. (I don't agree with this statement, our anterior to potsterior dimensions correlations were good)  Lateral dimension contrast was also challenging due to
peri-prostatic fat.  ARFI imaging is a promising low-cost, real-time imaging
modality that can compliment mpMRI for diagnosis, treatment planning and
management of PCa.

