\section*{Abstract}
Prostate cancer (PCa) is the most common non-cutaneous malignancy among men in
the United States and the second leading cause of cancer-related death.
Multi-parametric Magnetic Resonance Imaging (mpMRI) has gained recent
popularity to characterize PCa.  ARFI imaging has the potential to aid in PCa
diagnosis and management by evaluating the structural composition of prostate
zones and tumors based on their stiffness.  In this study, \invivo MR and ARFI
imaging datasets were compared to one another and with gross pathology
measurements made immediately after radical prostatectomy.  Images were
manually segmented in 3D Slicer to delineate the central gland (CG) and
prostate capsule, and 3D models were rendered to evaluate zonal anatomy
dimensions and volumes.  Both imaging modalities showed good correlation
between estimated organ volume and gross pathologic weights.  Ultrasound and MR
total prostate gland volumes were well-correlated (R$^2$ = \MRarfiVolTotalRsq),
but ARFI images yielded prostate volumes that were larger
(\MRarfiVolTotalMeanDiff\% $\pm$ \MRarfiVolTotalStdDiff\%) than MR images,
primarily due to overestimation of the lateral dimension
(\ARFImrTotalLatLatMeanPct $\pm$ \ARFImrTotalLatLatStdPct\%), while differences
in the other dimensions were less significant (\ARFImrTotalAntPostMeanPct $\pm$
\ARFImrTotalAntPostStdPct\%, anterior-to-posterior, and
\ARFImrTotalApexBaseMeanPct $\pm$ \ARFImrTotalApexBaseStdPct\%, apex-to-base).
The ARFI and MR CG volumes were also well-correlated (R$^2$ =
\MRarfiVolCentralRsq).  CG volume differences were attributed to ARFI
underestimation of the apex-to-base axis (\ARFImrCentralApexBaseMeanPct $\pm$
\ARFImrCentralApexBaseStdPct\%), and ARFI overestimation bias in the lateral
dimension (\ARFImrCentralLatLatMeanPct $\pm$ \ARFImrCentralLatLatStdPct\%).
ARFI imaging yielded prostate volumes and dimensions that were well-correlated
with MR T2WI estimates, with biases in the lateral dimension due to poor
contrast caused by peri-prostatic fat.  ARFI imaging is a promising low-cost,
portable, real-time imaging modality that can compliment mpMRI for diagnosis,
treatment planning and management of PCa.
