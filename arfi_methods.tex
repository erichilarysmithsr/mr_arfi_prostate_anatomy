\subsection{ARFI Imaging}
Experimental ARFI imaging data were acquired using a modified Siemens Acuson
SC2000\texttrademark~ultrasound scanner (Siemens Healthcare, Ultrasound Business Unit,
Mountain View, CA, USA) and the longitudinal array of an Acuson ER7B
transducer.  The ARFI imaging sequence was comprised of
standard B-mode ultrasonic imaging, or tracking beams, and pushing beams. For
each lateral location, two pre-push reference images were acquired, then three
300 cycle pushing pulses were transmitted in rapid succession, focused at 30
mm, 22.5 mm, and 15 mm, respectively, and finally the response of the tissue
was tracked for up to 6ms at a PRF of 8kHz. This pushing strategy is similar to
what has been published by Bercoff \etal~\cite{Bercoff2004}. The 30 mm and 22.5
mm foci pushing pulses were transmitted at 4.6 MHz with a F/2 geometry and the
15 mm focus pushing pulse was transmitted at 5.4 MHz with a F/2.35 geometry to
maintain the same beamwidth (0.67 mm) throughout the region of excitation. A
total of 82 lateral locations were interrogated to cover the 55 mm field of
view, translating 0.67 mm laterally per location.

For the tracking pulses, 16 parallel receive lines at 5.0 MHz were spaced to
observe both the on and off-axis response of the tissue to the pushing pulses.
Specifically, four lines were dedicated to tracking the on-axis displacement,
with all 4 beams located inside the beamwidth of the pushing pulses such that
the beam spacing was 0.17 mm. 

Displacement estimation was performed using a phase-shift estimator on the
beamformed in-phase and quadrature (IQ) data~\cite{Loupas95,pinton06}. The ARFI
data were then normalized as a function of depth to account for attenuation and
focal gain effects.  This normalization was performed using a displacement
profile measured in a homogeneous tissue-mimicking phantom, which was applied
to all displacements in the entire data set at each time step, and then
low-pass filtered with a cutoff-frequency of 0.8 mm$^{-1}$.

\centerline{\textcolor{red}{TO DO: INCLUDE TEXT AND FIGURE ABOUT THE TRANSDUCER ROTATION SETUP.}}
