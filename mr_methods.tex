\subsection{Study Inclusion Criteria}
Patients undergoing radical prostatectomy for biopsy-proven PCa treatment were
enrolled as study subjects in this IRB-approved (Duke IRB\# Pro00006458),
HIPAA-compliant study.  A total of 16 patients were recruited and enrolled in
this study.  Inclusion criteria were undergoing complete pelvic MRI with
endorectal coil for detection of prostate cancer, including multiplanar
T2-weighted anatomic imaging, as well as pre-operative ARFI imaging and radical
prostatectomy.  Patients with previous treatments of PCa or benign prostatic
hyperplasia (BPH), or anatomic anomalies of the rectum, were excluded from this
study.  All patients enrolled in this study provided written informed consent. 

\emph{CO-AUTHOR QUESTION: DO WE WANT TO INCLUDE A TABLE OF PATIENT DEMOGRAPHIC INFORMATION?}

\subsection{MR Imaging}
All MR imaging was performed on one of two 3.0 Tesla MR scanners (General
Electric HDx, GE Healthcare, Waukesha, WI;  Siemens Skyra, Siemens Healthcare,
Erlangan, Germany) using a single channel Medrad eCoil endorectal coil (Medrad,
Indianola, PA), as well as multi-channel surface coils.  Imaging sequences
included thin-section (3 mm section thickness) fast spin echo T2-weighted
images in the coronal, axial and sagittal planes.  Diffusion weighted images
were obtained using multiple b-values and calculation of ADC maps was also
performed.  Dynamic contrast enhanced MR sequences were obtained after
administration of a weight-based dose of extracellular MR contrast agent with
4-5 second temporal resolution for 5-6 minutes. (If we need it, can put in
table with full MR parameters) Prostates were radically removed using a da
Vinci Surgical System (Intuitive Surgical\textregistered, Inc., Sunnyvale, CA).
After exision, the prostates were weighed and measured, formalin fixed for at
least 24 hours without being cut, and then processed for whole mount histology.
