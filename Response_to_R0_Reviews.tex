\documentclass[10pt]{article}
\usepackage{graphicx}
\pagestyle{empty}
\topmargin=-0.75in
\oddsidemargin=0.0in
\textwidth=6.5in
\textheight=9.5in
\parindent=0.0in
\parskip=10pt

\newcommand{\exvivo}{\textit{ex vivo }}
\newcommand{\invivo}{\textit{in vivo }}
\newcommand{\invitro}{\textit{in vitro }}
\newcommand{\insitu}{\textit{in situ }}
\newcommand{\degree}{$^\circ$}
\newcommand{\isppa}{$I_\textrm{sppa}$}
\newcommand{\etal}{\textit{et al.}}
\newcommand{\attenunit}{dB$\cdot$cm$^{-1}$MHz$^{-1}$}



\begin{document}

\section*{Response to Reviews}

We thank both reviewers for their insight and comments on the original
submission of our manuscript.  Overall, well agree with all of the comments
that were provided, and we have made edits to our original manuscript that we
feel respond to all of this feedback.  We have provided a detailed response to
the review below, with the original comments italicized, and the responses in
normal text.  The revised manuscript has significant additions indicated with
bold text and removed text has been struck through.

Thank you, again, for your time and effort in this review process.

\textit{Reviewer: 1}

\textit{Comments to the Author}

\textit{This manuscripts describes exciting work demonstrating the potential relevance
of acoustic radiation force impulse (ARFI) and B-Mode ultrasound for
delineating prostate zonal anatomy dimensions and volumes in application to
prostate cancer diagnosis and treatment.  The current state of the art involves
MRI imaging, but an ultrasound-based complement would offer the benefits of
being low-cost, real-time and portable.  This work should be of great interest
to both the technically and clinically oriented UI readership.  Overall, the
manuscript is very well written, and the presented in vivo, clinical data
support the conclusions drawn. There are a few minor questions/comments for the
authors to consider before publication, which is recommended:}

\begin{itemize}

    \item \textit{An artistic rendering or other illustration of prostate
            anatomy would facilitate the readers' understanding of the
            description of prostate anatomy in section II.A.}

    We agree with this suggestion, and we have added Figure 1 to the manuscript
    that outlines the McNeals zonal anatomy of the prostaet in the coronal and
    sagittal views.

    \item \textit{Pg 5, lines 109-112, are redundant with other text previously
            written in the background.}

    We agree that this text is redundant, and we have removed it from this
    section of the manuscript.

    \item \textit{The methods section describes that diffusion weighted images
            (with ADC maps) and dynamic contrast enhanced MRI data were
            acquired in addition to the T2W images.  Were the the diffusion and
            contrast enhanced MRI data used in the MRI segmentation and/or
            analysis of MRI performance (as compared to ARFI).  It does not
            appear that they were.  Why not?  Also, what contrast agent was
            used?}

    The Methods section of the manuscript describes the comprehensive
    multiparametric MRI protocol that is used at our institution for imaging of
    the prostate gland.  It does include T2-weighted (T2W) imaging for anatomy
    as well as diffusion weighted images (DWI) and dynamic contrast enhanced
    imaging (DCE-MRI) for functional information as well as lesion
    characterization.  The T2W images are the highest resolution images in our
    dataset and therefore, would give the most accurate information for
    prostate size as compared with ARFI.  The DWI and DCE-MRI images are used
    in conjunction with the T2W images for lesion characterization, but due to
    their lower resolution, would likely not add much information to the T2W
    images for the purposes of prostate size and segmentation.  Additional
    research ongoing in our group does focus on lesion segmentation, and for
    this application, information from all three sequences are used in
    conjunction. All MRIs were performed using a standard extracellular MR
    contrast agent (Magnevist, Bayer Pharma AG). 

    To clarify this point for the reader, we have added the following text to
    Section III(c) of the manuscript:

    \textbf{While the DWI and DCE-MRI images were used in conjuntion with the T2W
images for lesion characterization, the DWI and DCE-MRI images were lower
resolution and do not add much information to the T2W images for the purposes
of delineating prostate size and segmentation.
}

    The contrast agent that was used has also been added to the manuscript.

    \item \textit{Please describe what is meant by the "longitudinal array" (pg
            6 line 151) and specify that the ER7B is an endorectal transducer.
            This was stated in the associated figure caption but not in the
            text.  Also, please state more clearly in the text that ARFI
            imaging was performed endorectally.}

    We have modified the text describing the ER7B array as follows:

    \textbf{\ldots side-fire array of an Acuson ER7B endorectal transducer.}

    We have also emphasized that the imaging was performed endorectally by
    modifying the following statement in the same section:

    \textbf{3D volumetric imaging data was acquired endorectally\ldots}

    \item \textit{Why were the tracking beams focused (@ 60 mm) twice as deep
            as the pushing beams (@ 30 mm and above)?  It would seem that the
            radiation force would not propagate this far below the pushing
            focal position. (pg 6 lines 163 and 154, respectively).}

    The tracking beam transmit was focused at 60 mm to insonify a large lateral
    field of view and observe the off-axis shear wave propagation with the
    remaining 12 parallel receive beams that were not utilized in this work
    (lines 160-161). For each track beam, dynamic receive and receive aperture
    growth with a F/0.5 configuration were used to minimize the effect of the
    deep transmit focus.  Since only the ARFI imaging vectors were presented in
    this work--not the shear wave data--we have not ammended the manuscript
    text to avoid confusing the reader with details not relevant to the data
    presented in this manuscript.

    \item \textit{For the displacement normalization factor derived in phantom
            materials (pg 7 lines 176-177), what were the acoustic and
            mechanical properties of the phantom, and were these well matched
            to the expected properties of prostate tissue?  How much are these
            properties expected to vary from individual to individual, with or
            without prostate cancer?  What would the impact of using of an
            incorrect normalization factor be on the overall ARFI results?  The
            authors should consider adding a discussion of the potential impact
            of the assumed normalization factor to the limitations sections.}

    This is a great comment, and one that we have considered internally during
    the development of our methodology for this study.  The normalization
    profile for ARFI images is similar in concept to time gain compensation
    (TGC) for B-mode images. It is used to apply gain throughout the image to
    provide a more uniform appearance of the image through depth, which is
    necessary to account for tissue attenuation and focal gain. The
    normalization does not change the raw image data since it is applying
    digital gain. Additionally, when a reader views the images, gain can be
    added or removed as desired to further improve the image quality.
    Therefore, on a patient-by-patient basis, the normalization profile can be
    adjusted to account for differences in each individual to reduce the impact
    of having an incorrect normalization profile.

    Specifically, for this work, two calibrated CIRS (Norfolk, VA) elasticity
    phantoms were used to determine the normalization factor; the background
    stiffness of both materials was nominally 10 kPa Young's modulus, and the
    acoustic attenuation of the two phantoms were 0.49 and 0.72 dB/cm/MHz,
    respectively. The displacement-through-time profiles from each phantom were
    averaged together to approximate a 10 kPa Young's modulus with a 0.6
    dB/cm/MHz attenuation. The phantom stiffness is slightly lower than the
    12.3 kPa Young's modulus reported as healthy peripheral zone tissue [1].
    The attenuation coefficient for prostatic tissue has been reported as
    0.54-1.04 dB/cm/MHz [2], with additional support that the attenuation is at
    the lower end of that range [3]. Thus, the normalization profile closely
    approximates the properties of healthy peripheral zone prostatic tissue,
    but the transition and central zones as well as pathology such as cancer,
    BPH, or atrophy can all potentially have different mechanical and
    acoustical properties.

    [1] Zhai, Liang, et al. ``Correlation between SWEI and ARFI image findings
    in ex vivo human prostates.'' Ultrasonics Symposium (IUS), 2009 IEEE
    International. IEEE, 2009.

    [2] K.J Parker, S.R Huang, R.M Lerner et al. Elastic and ultrasonic
    properties of the prostate IEEE Ultrason Sympos Proc, 2 (1993), pp.
    1035–1038

    [3] A.E Worthington, J Trachtenberg, M.D Sherar, Ultrasound properties of
    human prostate tissue during heating, Ultrasound in Medicine \& Biology,
    Volume 28, Issue 10, October 2002, Pages 1311-1318.

    The following text has been added to Section III(C):

    \textbf{The ARFI data were normalized as a function of depth to account for   
attenuation and focal gain effects to ensure that ARFI image brightness
differences were more directly related to stiffness differences.  This                
normalization was performed using a displacement profiles measured in two            
homogeneous, elastic tissue-mimicking phantoms (CIRS, Norfolk, VA) with a          
Youngs modulus of 10.0 kPa and acoustic attenuations of 0.49 and 0.72 dB/cm/MHz.        
The mean displacement profiles from each phantom were averaged together to
achieve a mean attenuation of 0.6 dB/cm/MHz, which is within the measured
range of attenuation in healthy peripheral zone prostate
tissue~\cite{Parker1993}.  This mean displacement profile was low-pass filtered
with a cutoff-frequency of 0.8 mm$^{-1}$ and applied to all displacements in
the entire dataset at each time step.}


    \item \textit{Pg 8 lines 205-214 describe that both B-Mode (for prostate
            capsule) and ARFI (for CG) data were used for segmentation.  First,
            how were these data combined?  Second, the use of B-Mode
            segmentation in addition to ARFI is not clearly specified in the
            titles, text and figures.  For example, in Table III, was the total
            volume calculation a result of both ARFI and B-Mode segmentations?
            In that case, the title and column labels describing "ARFI" total
            volume are confusing because they imply that the B-Mode
            segmentation data were not used.  The same is true for the other
            tables, figures, text, and the main Title of the manuscript.  If I
            understand correctly that B-Mode data were important to the overall
            total volume segmentation, then it would be appropriate to better
            clarify this point.  The point is currently confused by the main
            manuscript title, text and figures that generally refer to "ARFI"
            performance.  I believe the authors try to indicate that both
            B-Mode and ARFI data are used for the segmentation by referring to
            the "ultrasound" imaging performance, as in "… with a mean
            overestimation of of 16.82 ± 22.45\% by ultrasound imaging compared
            to MR volumes (Figure 6(b))." (pg 9 line 225), but this reader did
            not immediately appreciate that "ultrasound" referred to combined
            B-mode and ARFI segmentation as opposed to ARFI alone, and the
            "ARFI" labels throughout the rest of the text, tables, titles, and
            figures further confused this point.}

    This comment is greatly appreciated since the use of both B-mode and ARFI
    imaging for each segmentation was an important part of our methodology, and
    we (1) did not do enough to emphasize this point, and (2) confused matters
    by refering to `ARFI' exclusively in the results.  With respect to how the
    data were combined, we have added the following text to the manuscript
    (Section III(E.2)):

    \textbf{Since both B-mode and ARFI imaging data were acquired concurrently using the same probe and sequences, the datasets are perfectly spatially coregistered and could be loaded concurrently into 3D Slicer in the same voxel space, achieving perfect spatial registration of segmentations utilizing both B-mode and ARFI image features.}


    We also agree that our exclusive use of 'ARFI' for many of the tables,
    figures, manuscript title, and text was confusing, so we have updated all
    references to appropriate include B-mode when appropriate (not delineated
    here in the response due to the number of instances that were edited, but
    all manuscript changes are indicated in \textbf{bold} typeface).  The title
    of the manuscript has also been updated to reflect the importance of B-mode
    imaging in this process, and the abstract and conclusions have also been
    explicitely updated to include the fact that B-mode and ARFI imaging data
    were used in this study.

    \item \textit{How many readers performed the B-Mode and ARFI segmentation
            per image? per entire study data set? Were the readers trained, and
            if so, how?  Did different readers segment the B-Mode and ARFI data
            sets?  Did the same radiologist who segmented the MRI images also
            segment the B-Mode and ARFI data?}

    All of the B-mode and ARFI images were segemented by two individuals (ZAM,
    TJG), and reviewed by a single, experienced ARFI imaging research (KRN).
    There was no formal training in reading the ARFI images since this work is
    our first effort to analyze our ability to evaluate our accuracy in
    identifying structures in ARFI prostate images.  The same individuals
    segmented both the B-mode images (for the capsules) and ARFI images (for
    the central glands).  All of the MR images were segmented by an experienced
    radiology fellow (CK) and reviewed by an experienced prostate radiologist
    (RG).  We have added the following text to the manuscript to delineate the
    readers / image segmenters for each part of the study (Section III(E.1)):

    Axial MR T2WI images were manually segmented by a \textbf{single
        radiologist (CK)} using the smooth polygon tool in ITK-SNAP\ldots

    \textbf{All MR image segmentations were ultimately reviewed by an
        experienced prostate radiologist (RTG).}

    And Section III(E.2):

    Sagittal B-mode and ARFI ultrasound image stacks were segmented in 3D
    Slicer \textbf{by two researchers (TJG \& ZAM)}\ldots

    \textbf{All B-mode and ARFI image segmentations were reviewed by an
        experienced ARFI imaging researcher (KRN).}


    \item \textit{How would limits to ARFI resolution impact the results?}

    TO DO

    \item \textit{The authors explain that gross pathology weights and axis
            measurements were affected by the presence of peri-prostatic
            tissue.  Why could the peri-prostatic tissue not be differentiated
            from the prostatic tissue in histology after processing?  Wouldn't
            this enable better gold standard axis measurements?}

    We agree that very careful dissection of the gross organ from adjacent
    peri-prostatic tissue could be achieved, but to not disrupt the clinical
    pathologic processing of these samples, we could not specify specific
    pathology assisstants to process these specimens, so there was some
    inherent variability associated with the time and effort that each
    assistant dedicated to removing this peri-prostatic tissue.  We felt that
    it was important to delineate this as a source of variability among our
    pathology specimens, though it is expected to be a higher-order source of
    error.

    \item \textit{Further to the point above, if the prostatic tissue axes
            could be measured after histological processing, could you perform
            an ROC analysis to assess ARFI/B-Mode versus MRI performance for
            axis measurement?}

    As is also detailed in the response to comment 2 by the second reviewer, we
    are actively pursuing correlation of our B-mode/ARFI and MR images with
    whole mount histology, which includes rigid and diffeomorphic image
    registration processes.  Given the that the methods and validation
    associated with the more spatially-refined whole mount histology analysis
    and diffeomorphic image registration is nontrivial, we have decided to
    present that material in a dedicated manuscript that will allow us to
    present our validation of that methodology.

    \item \textit{pg 12, line 329, change ``compliment'' to ``complement''.}

    Thank you for catching this grammatical error.  It actually existed in both
    the abstract and conclusion sections, and it has been corrected in both
    sections.

    \item \textit{Note that in my version of the manuscript, the figures and
            the captions written below them are blurry and difficult to read.
            This may just be a formatting issue with the automatic reviewer
            support system, but it is worth confirming that figures and their
            captions will be clear for publication.}

    High resolution images were uploaded for the review process, but their
    quality was definitely degraded during the rendering of the review PDF.  We
    will make sure that all images are propoerly reproduced at full resolution
    in new PDFs that are rendered of this manuscript.

\end{itemize}

\textit{Reviewer: 2}

\textit{Comments to the Author}

\textit{This is a nice study of ARFI imaging of the prostate that compares ARFI images
to MRI. Three areas of the paper could be improved:}

\begin{itemize}
    \item \textit{The ellipsoidal prostate volume estimate from pathology could be
        improved by using the prostate segmentation on each of the whole mount
        pathology slides to align the data to the MRI prostate surface. The
        volume comparison is further complicated by the shrinkage of the
        prostate after fixation.}

   Our ellipsoidal approximation of the prostate volume is a rough
   approximation from the surgical pathology reports, but we felt it was our
   best metric to approximate prostate size for the following reasons:

   \begin{itemize}
        \item The gross surgical pathology mesaurements used to make the
            ellipsoidal volume approximation were done after peri-prostatic
            tissue was dissected away, but before formalin fixation, avoiding
            the complication of tissue shrinkage associated with formalin
            fixation, which our pathologists estimated could be as much as 30\%
            of the prostate volume.
        \item Whole mount histology slides were explored for this purpose, but
            they suffer from post-fixation contraction, $\sim$ 3 mm slice
            thickiness, and slice tearing and folding that compromise accurate
            cross-sectional area representations on a single slide.
            Additionally, the whole mount histology slides require some degree
            of image registration due to their arbitrary orientation on the
            slides, adding another source of error.
    \end{itemize}

    Given the reasons above, we decided to use the tri-axial measurements in
    gross pathologic evaluation as our closest approximation of the glands
    total volume as it would have been during MR and ultrasound imaging.  We
    have added text to Section II.B that details this choice for estimating our
    prostate volumes:

    \textbf{An alternative approach to estimating the prostate volume was to reconstruct
the prostate capsule from serial whole mount histology slides of the prostate.
This approach, however, was not chosen because the formalin fixation process
can cause 0-25\% contraction of the tissue, the $\sim$3 mm slice thickness
introduces error, and tearing / folding of tissue on the whole mount slides
introduce additional artifacts that need to be compensated for in the volume
estimation process.  Finally, differences in the tissue orientation on each
whole mount slide would require an image registration process to align all of
the slides, introducing another source of error in the process.
}

    \item \textit{Image registration. The three orthogonal dimensions of the
            images do not mean much. The slice thickness in MRI causes
            significant errors along this measurement axis, while the
            lateral-to-lateral dimension in ARFI seems to be enlarged by the
            TRUS deformation (this is not mentioned in the discussion, but
            seems to be fairly obvious from Fig. 2 and 5, for example). A
            Jaccard index type of comparison would be preferable after rigid or
            possibly volume-preserving deformable registration.}

    TO DO

    \item \textit{discussion of prior work is inadequate. Volumetric
            acquisition and comparison of mechanical imaging with B-mode and
            MRI done before for vibro-elastography if not for ARFI and that
            work should be cited and discussed.}

    We did mistakenly omit very relevant work that has been done with
    vibro-elastography and MR imaging in the prostate, and we have included
    this work in the manuscript's Introduction:

    \textbf{Similar work has been done with other ultrasonic imaging modalities that strive
to characterize the viscoelastic properties of the prostate in comparison to MR
imaging, as has been done by Mahdavi \etal~\cite{Mahdavi2011}, where
vibro-elastography has been developed to visualize and segment the prostate in
synergy with B-mode ultrasound, with MRI used as the imaging gold standard.
}

\end{itemize}

\end{document}
