\documentclass[10pt]{article}
\usepackage{graphicx}
\pagestyle{empty}
\topmargin=-0.75in
\oddsidemargin=0.0in
\textwidth=6.5in
\textheight=9.5in
\parindent=0.0in
\parskip=10pt

\begin{document}

\section*{Response to Reviews}

We thank both reviewers for their insight and comments on the original
submission of our manuscript.  Overall, well agree with all of the comments
that were provided, and we have made edits to our original manuscript that we
feel respond to all of this feedback.  We have provided a detailed response to
the review below, with the original comments italicized, and the responses in
normal text.

Thank you, again, for your time and effort in this review process.

\textit{Reviewer: 1}

\textit{Comments to the Author}

\textit{This manuscripts describes exciting work demonstrating the potential relevance
of acoustic radiation force impulse (ARFI) and B-Mode ultrasound for
delineating prostate zonal anatomy dimensions and volumes in application to
prostate cancer diagnosis and treatment.  The current state of the art involves
MRI imaging, but an ultrasound-based complement would offer the benefits of
being low-cost, real-time and portable.  This work should be of great interest
to both the technically and clinically oriented UI readership.  Overall, the
manuscript is very well written, and the presented in vivo, clinical data
support the conclusions drawn. There are a few minor questions/comments for the
authors to consider before publication, which is recommended:}

\begin{itemize}

    \item \textit{An artistic rendering or other illustration of prostate
            anatomy would facilitate the readers' understanding of the
            description of prostate anatomy in section II.A.}

    We agree with this suggestion, and we have added Figure 1 to the manuscript
    that outlines the McNeals zonal anatomy of the prostaet in the coronal and
    sagittal views.

    \item \textit{Pg 5, lines 109-112, are redundant with other text previously
            written in the background.}

    \item \textit{The methods section describes that diffusion weighted images
            (with ADC maps) and dynamic contrast enhanced MRI data were
            acquired in addition to the T2W images.  Were the the diffusion and
            contrast enhanced MRI data used in the MRI segmentation and/or
            analysis of MRI performance (as compared to ARFI).  It does not
            appear that they were.  Why not?  Also, what contrast agent was
            used?}

    \item \textit{Please describe what is meant by the "longitudinal array" (pg
            6 line 151) and specify that the ER7B is an endorectal transducer.
            This was stated in the associated figure caption but not in the
            text.  Also, please state more clearly in the text that ARFI
            imaging was performed endorectally.}

    \item \textit{Why were the tracking beams focused (@ 60 mm) twice as deep
            as the pushing beams (@ 30 mm and above)?  It would seem that the
            radiation force would not propagate this far below the pushing
            focal position. (pg 6 lines 163 and 154, respectively).}

    \item \textit{For the displacement normalization factor derived in phantom
            materials (pg 7 lines 176-177), what were the acoustic and
            mechanical properties of the phantom, and were these well matched
            to the expected properties of prostate tissue?  How much are these
            properties expected to vary from individual to individual, with or
            without prostate cancer?  What would the impact of using of an
            incorrect normalization factor be on the overall ARFI results?  The
            authors should consider adding a discussion of the potential impact
            of the assumed normalization factor to the limitations sections.}

    \item \textit{Pg 8 lines 205-214 describe that both B-Mode (for prostate
            capsule) and ARFI (for CG) data were used for segmentation.  First,
            how were these data combined?  Second, the use of B-Mode
            segmentation in addition to ARFI is not clearly specified in the
            titles, text and figures.  For example, in Table III, was the total
            volume calculation a result of both ARFI and B-Mode segmentations?
            In that case, the title and column labels describing "ARFI" total
            volume are confusing because they imply that the B-Mode
            segmentation data were not used.  The same is true for the other
            tables, figures, text, and the main Title of the manuscript.  If I
            understand correctly that B-Mode data were important to the overall
            total volume segmentation, then it would be appropriate to better
            clarify this point.  The point is currently confused by the main
            manuscript title, text and figures that generally refer to "ARFI"
            performance.  I believe the authors try to indicate that both
            B-Mode and ARFI data are used for the segmentation by referring to
            the "ultrasound" imaging performance, as in "… with a mean
            overestimation of of 16.82 ± 22.45\% by ultrasound imaging compared
            to MR volumes (Figure 6(b))." (pg 9 line 225), but this reader did
            not immediately appreciate that "ultrasound" referred to combined
            B-mode and ARFI segmentation as opposed to ARFI alone, and the
            "ARFI" labels throughout the rest of the text, tables, titles, and
            figures further confused this point.}

    \item \textit{How many readers performed the B-Mode and ARFI segmentation
            per image? per entire study data set? Were the readers trained, and
            if so, how?  Did different readers segment the B-Mode and ARFI data
            sets?  Did the same radiologist who segmented the MRI images also
            segment the B-Mode and ARFI data?}

    \item \textit{How would limits to ARFI resolution impact the results?}

    \item \textit{The authors explain that gross pathology weights and axis
            measurements were affected by the presence of peri-prostatic
            tissue.  Why could the peri-prostatic tissue not be differentiated
            from the prostatic tissue in histology after processing?  Wouldn't
            this enable better gold standard axis measurements?}

    \item \textit{Further to the point above, if the prostatic tissue axes
            could be measured after histological processing, could you perform
            an ROC analysis to assess ARFI/B-Mode versus MRI performance for
            axis measurement?}

    \item \textit{pg 12, line 329, change "compliment" to "complement".}

    \item \textit{Note that in my version of the manuscript, the figures and
            the captions written below them are blurry and difficult to read.
            This may just be a formatting issue with the automatic reviewer
            support system, but it is worth confirming that figures and their
            captions will be clear for publication.}

\end{itemize}

\textit{Reviewer: 2}

\textit{Comments to the Author}

\textit{This is a nice study of ARFI imaging of the prostate that compares ARFI images
to MRI. Three areas of the paper could be improved:}

\begin{itemize}
    \item \textit{the ellipsoidal prostate volume estimate from pathology could be
        improved by using the prostate segmentation on each of the whole mount
        pathology slides to align the data to the MRI prostate surface. The
        volume comparison is further complicated by the shrinkage of the
        prostate after fixation.}

    \item \textit{Image registration. The three orthogonal dimensions of the
            images do not mean much. The slice thickness in MRI causes
            significant errors along this measurement axis, while the
            lateral-to-lateral dimension in ARFI seems to be enlarged by the
            TRUS deformation (this is not mentioned in the discussion, but
            seems to be fairly obvious from Fig. 2 and 5, for example). A
            Jaccard index type of comparison would be preferable after rigid or
            possibly volume-preserving deformable registration.}

    \item \textit{discussion of prior work is inadequate. Volumetric
            acquisition and comparison of mechanical imaging with B-mode and
            MRI done before for vibro-elastography if not for ARFI and that
            work should be cited and discussed.}
\end{itemize}

\end{document}
